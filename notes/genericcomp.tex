\section{Authenticated Encryption from SE and MACs}
\label{sec:genericcomp}
Let $\SE=(\SE.\kg,\SE.\enc,\SE.\dec)$ be a symmetric encryption scheme and $\MA=(\MA.\kg,\mtag,\ver)$ be a message authentication scheme. In this section, we combine these two primitives in different ways to construct an authenticated encryption scheme $\SE'=(\kg,\enc,\dec)$, and we describe the security guarantees of each construction. 
\begin{figure}[h]
	\centering
	\fpage{.12}{
		\underline{$\kg$}\\[1pt]
		$K_1\getsr \SE.\kg$\\
		$K_2\getsr \MA.\kg$\\
		Ret $(K_1,K_2)$
	}
	\fpage{.16}{
		\underline{$\enc(K,M)$}\\[1pt]
		$(K_1,K_2)\gets K$\\
		$C\gets \SE.\enc(K_1,M)$\\
		$T\gets \mtag(K_2,M)$\\
		Ret $(C,T)$
	}
	\fpage{.18}{
		\underline{$\dec(K,(C,T))$}\\[1pt]
		$(K_1,K_2)\gets K$\\
		$M\gets \SE.\dec(K_1,C)$\\
		$T'\gets \mtag(K_2,M)$\\
		If $T'\neq T$ then Ret $\bot$\\
		Ret $M$
	}
	\caption{The Encrypt-and-Mac composition. The ciphertext and tag are both computed on the original message.}
\end{figure}
\begin{figure}[h]
	\centering
	\fpage{.12}{
		\underline{$\kg$}\\[1pt]
		$K_1\getsr \SE.\kg$\\
		$K_2\getsr \MA.\kg$\\
		Ret $(K_1,K_2)$
	}
	\fpage{.2}{
		\underline{$\enc(K,M)$}\\[1pt]
		$(K_1,K_2)\gets K$\\
		$T\gets \mtag(K_2,M)$\\
		$C\gets \SE.\enc(K_1,M\concat T)$\\
		Ret $C$
	}
	\fpage{.22}{
		\underline{$\dec(K,C)$}\\[1pt]
		$(K_1,K_2)\gets K$\\
		$M\concat T\gets \SE.\dec(K_1,C)$\\
		\scribenote{fix the concat notation}\\
		$T'\gets \mtag(K_2,M)$\\
		If $T'\neq T$ then Ret $\bot$\\
		Ret $M$
	}
	\caption{The Mac-then-Encrypt composition. The tag is computed on the original message, then the message and the tag are encrypted together.}
\end{figure}
\begin{figure}[h]
	\centering
	\fpage{.12}{
		\underline{$\kg$}\\[1pt]
		$K_1\getsr \SE.\kg$\\
		$K_2\getsr \MA.\kg$\\
		Ret $(K_1,K_2)$
	}
	\fpage{.16}{
		\underline{$\enc(K,M)$}\\[1pt]
		$(K_1,K_2)\gets K$\\
		$C\gets \SE.\enc(K_1,M)$\\
		$T\gets \mtag(K_2,C)$\\
		Ret $(C,T)$
	}
	\fpage{.18}{
		\underline{$\dec(K,(C,T))$}\\[1pt]
		$(K_1,K_2)\gets K$\\
		$M\gets \SE.\dec(K_1,C)$\\
		$T'\gets \mtag(K_2,C)$\\
		If $T'\neq T$ then Ret $\bot$\\
		Ret $M$
	}
	\caption{The Encrypt-then-Mac composition. The ciphertext is computed on the original message, then the tag is computed on the ciphertext.}
\end{figure}

If we assume that $\SE$ is $\ROR$ secure and $\MA$ is $\UFCMA$ secure, then we can say that the Encrypt-then-Mac composition of them is $\RORCCA$ secure, which we prove in Theorem~\ref{thm:rorcca-etm}. However, we cannot say the same for the Encrypt-and-Mac or Mac-then-Encrypt compositions in general. \scribenote{Give counter examples for EaM and MtE?}

\begin{theorem}
Let $\SE=(\SE.\kg,\SE.\enc,\SE.\dec)$ be an SE scheme and $\MA=(\MA.\kg,\MA.\mtag,\MA.\ver)$ be a MAC and $\EtM=(\EtM.\kg,\EtM.\enc,\EtM.\dec)$ be the Encrypt-then-Mac
scheme built from them.  Let $\advA$ be an $\RORCCA_\EtM$-adversary making at most
$q$ queries. Then we give a $\ROR_\SE$-adversary $\advB_{\textrm{se}}$ 
and a $\UFCMA_\MA$-adversary  $\advB_\textrm{mac}$ 
such that
\bnm
  \AdvRORCCA{\EtM}{\advA} \le \AdvROR{\SE}{\advB_\textrm{se}} + 2\cdotsm\AdvUFCMA{\MA}{\advB_{\textrm{mac}}} \;.
\enm
Adversaries $\advB_\textrm{se}$ and $\advB_{\textrm{mac}}$ each run in time that of 
$\advA$ plus small overhead and each make at most $q$ queries.
\label{thm:rorcca-etm}
\end{theorem}

\begin{figure}[t]
\hfpagess{.23}{.15}{
	\underline{$\G0$}\\[1pt]
	$K_1 \getsr \SE.\kg$\\
	$K_2 \getsr \MA.\kg$\\
	$b' \getsr \advA^{\EncOracle,\DecOracle}$\\
	Ret $b'$\medskip
	
	\underline{$\EncOracle(M)$}\\
	$C \getsr \SE.\enc(K_1,M)$\\
	$T \gets \MA.\mtag(K_2,C)$\\
	$\calC \gets \calC \cup \{(C,T)\}$\\
	Ret $(C,T)$\medskip
	
	\underline{$\DecOracle((C,T))$}\\
	If $(C,T) \in \calC$ then \\
	\myInd Ret $\bot$\\
	$M\gets \SE.\dec(K_1,C)$\\
	If $\MA.\ver(K_2,C,T)=1$ then \\
	\myInd Ret $M$\\
	Ret $\bot$
}{
	\underline{$\G1$}\\[1pt]
	$b' \getsr \advA^{\EncOracle,\DecOracle}$\\
	Ret $b'$\medskip
	
	\underline{$\EncOracle(M)$}\\
	$C \getsr \bits^{\ctxtlen(|M|)}$\\
	Ret $C$\medskip
	
	\underline{$\DecOracle(C)$}\\
	Ret $\bot$
}	
\end{figure}
\begin{proof}
	\scribenote{TODO}
\end{proof}