\section{Authenticated Encryption from SE and MACs}
\label{sec:genericcomp}
Let $\SE=(\SE.\kg,\SE.\enc,\SE.\dec)$ be a symmetric encryption scheme and $\MA=(\MA.\kg,\mtag,\ver)$ be a message authentication scheme. In this section, we combine these two primitives in different ways to construct an authenticated encryption scheme $\SE'=(\kg,\enc,\dec)$, and we describe the security guarantees of each construction. 
\begin{figure}[h]
	\centering
	\fpage{.12}{
		\underline{$\kg$}\\[1pt]
		$K_1\getsr \SE.\kg$\\
		$K_2\getsr \MA.\kg$\\
		Ret $(K_1,K_2)$
	}
	\fpage{.16}{
		\underline{$\enc(K,M)$}\\[1pt]
		$(K_1,K_2)\gets K$\\
		$C\gets \SE.\enc(K_1,M)$\\
		$T\gets \mtag(K_2,M)$\\
		Ret $(C,T)$
	}
	\fpage{.18}{
		\underline{$\dec(K,(C,T))$}\\[1pt]
		$(K_1,K_2)\gets K$\\
		$M\gets \SE.\dec(K_1,C)$\\
		$T'\gets \mtag(K_2,M)$\\
		If $T'\neq T$ then Ret $\bot$\\
		Ret $M$
	}
	\caption{The Encrypt-and-Mac composition. The ciphertext and tag are both computed on the original message.}
\end{figure}
\begin{figure}[h]
	\centering
	\fpage{.12}{
		\underline{$\kg$}\\[1pt]
		$K_1\getsr \SE.\kg$\\
		$K_2\getsr \MA.\kg$\\
		Ret $(K_1,K_2)$
	}
	\fpage{.2}{
		\underline{$\enc(K,M)$}\\[1pt]
		$(K_1,K_2)\gets K$\\
		$T\gets \mtag(K_2,M)$\\
		$C\gets \SE.\enc(K_1,M\concat T)$\\
		Ret $C$
	}
	\fpage{.22}{
		\underline{$\dec(K,C)$}\\[1pt]
		$(K_1,K_2)\gets K$\\
		$M\concat T\gets \SE.\dec(K_1,C)$\\
		\scribenote{fix the concat notation}\\
		$T'\gets \mtag(K_2,M)$\\
		If $T'\neq T$ then Ret $\bot$\\
		Ret $M$
	}
	\caption{The Mac-then-Encrypt composition. The tag is computed on the original message, then the message and the tag are encrypted together.}
\end{figure}
\begin{figure}[h]
	\centering
	\fpage{.12}{
		\underline{$\kg$}\\[1pt]
		$K_1\getsr \SE.\kg$\\
		$K_2\getsr \MA.\kg$\\
		Ret $(K_1,K_2)$
	}
	\fpage{.16}{
		\underline{$\enc(K,M)$}\\[1pt]
		$(K_1,K_2)\gets K$\\
		$C\gets \SE.\enc(K_1,M)$\\
		$T\gets \mtag(K_2,C)$\\
		Ret $(C,T)$
	}
	\fpage{.18}{
		\underline{$\dec(K,(C,T))$}\\[1pt]
		$(K_1,K_2)\gets K$\\
		$M\gets \SE.\dec(K_1,C)$\\
		$T'\gets \mtag(K_2,C)$\\
		If $T'\neq T$ then Ret $\bot$\\
		Ret $M$
	}
	\caption{The Encrypt-then-Mac composition. The ciphertext is computed on the original message, then the tag is computed on the ciphertext.}
\end{figure}

If we assume that $\SE$ is $\ROR$ secure and $\MA$ is $\UFCMA$ secure, then we can say that the Encrypt-then-Mac composition of them is $\RORCCA$ secure, which we prove in Theorem~\ref{thm:rorcca-etm}. However, we cannot say the same for the Encrypt-and-Mac or Mac-then-Encrypt compositions in general. \scribenote{Give counter examples for EaM and MtE?}

\begin{theorem}
Let $\SE=(\SE.\kg,\SE.\enc,\SE.\dec)$ be an SE scheme and $\MA=(\MA.\kg,\MA.\mtag,\MA.\ver)$ be a MAC with $\MA.\mtag$ output length $n$ and $\EtM=(\EtM.\kg,\EtM.\enc,\EtM.\dec)$ be the Encrypt-then-Mac
scheme built from them.  Let $\advA$ be an $\RORCCA_\EtM$-adversary making at most
$q$ queries. Then we give a $\ROR_\SE$-adversary $\advB_{\textrm{se}}$ 
and a $\UFCMA_\MA$-adversary  $\advB_\textrm{mac}$ 
such that
\bnm
  \AdvRORCCA{\EtM}{\advA} \le \AdvROR{\SE}{\advB_\textrm{se}} + \AdvPRF{\MA}{\advB_{\textrm{mac}}} + \frac{1}{2^n} \;.
\enm
Adversaries $\advB_\textrm{se}$ and $\advB_{\textrm{mac}}$ each run in time that of 
$\advA$ plus small overhead and each make at most $q$ queries.
\label{thm:rorcca-etm}
\end{theorem}

\begin{figure}[t]
	\centering
\hfpagesss{.23}{.2}{.2}{
	\underline{$\advB_{\textrm{se}}^\EncOracle$}\\[1pt]
	$K_2 \getsr \MA.\kg$\\
	$b' \getsr \advA^{\EncSim}$\\
	Ret $b'$\medskip
	
	\underline{$\EncSim(M)$}\\
	$C \gets \EncOracle(M)$\\
	$T \gets \MA.\mtag(K_2,C)$\\
	Ret $(C,T)$\medskip
}{
	\underline{$\advB_{\textrm{mac}}^{\Fn}$}\\[1pt]
	$K \getsr \SE.\kg$\\
	$b'\gets 0$\\
	$\advC^{\EncSim,\DecSim}$\\
	Ret $b'$\medskip
}{
	\underline{$\EncSim(M)$}\\
	$C \getsr \SE.\enc(K,M)$\\
	$T \gets \Fn(C)$\\
	$\mathcal{S} \gets \mathcal{S}\union \{(C,T)\}$\\
	Ret $(C,T)$\medskip
	
	\underline{$\DecSim(C,T)$}\\
	If $(C,T)\in\mathcal{S}$ then\\
	\myInd Ret $\bot$\\
	$M \gets\SE.\dec(K,C)$\\
	If $\Fn(C)=T$ then\\
	\myInd $b'\gets 1$\\
	\myInd Ret $M$\\
	Else Ret $\bot$
}	
\caption{Adversaries used for proof of Theorem~\ref{thm:rorcca-etm}.}
\label{fig:proof-etm-rorcca}
\end{figure}
\begin{proof}
	As we saw in Theorem \scribenote{Cite from previous section once it's done},
	\bnm
	\AdvRORCCA{\EtM}{\advA}\leq \AdvROR{\EtM}{\advB} + \AdvCTXT{\EtM}{\advC}
	\enm
	First, we know that 
	\bnm
	\AdvROR{\EtM}{\advB} \leq \AdvROR{\SE}{\advB_\textrm{se}}\;.
	\enm
	Given a $\ROR$ distinguisher for $\advB$ for $\EtM$, we can build a $\ROR$ adversary $\advB_\textrm{se}$ for $\SE$. 
	
	Also,
	\bnm
	\AdvCTXT{\EtM}{\advC} \leq \AdvPRF{\MA}{\advB_{\textrm{mac}}}+\frac{1}{2^n}\;.
	\enm
	
	If the output of the $\Fn$ oracle is real, then $\advB_{\textrm{mac}}$ returns 1 if and only if adversary $\advC$ is able to forge a valid ciphertext, so
	\bnm
	\Prob{\PRF1_\MA^{\advB_{\textrm{mac}}}\Rightarrow 1}=\Prob{\CTXT_{\EtM}^{\advC}}\;.
	\enm
	If the output of the $\Fn$ oracle is random, then the probability that $\advC$ forges the tag successfully is $1/2^n$, if $n$ is the length of the tag. 
	\bnm
	\Prob{\PRF0_\MA^{\advB_{\textrm{mac}}}\Rightarrow 1}\leq \frac{1}{2^n}\;.
	\enm
	Combining these, we have
	\begin{align*}
	\AdvPRF{\MA}{\advB_{\textrm{mac}}}&=\left|\Prob{\PRF1_\MA^{\advB_{\textrm{mac}}}\Rightarrow 1}-\Prob{\PRF0_\MA^{\advB_{\textrm{mac}}}\Rightarrow 1}\right|\\
	&\geq \AdvCTXT{\EtM}{\advC}-\frac{1}{2^n}\;.
	\end{align*}
\end{proof}