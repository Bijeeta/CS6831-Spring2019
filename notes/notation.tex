%%%%%%%%%%%%%%%%%%%%%%%%%%%%%%%%%%%%%%%%%%%%%%%%%%%%%%%%%%%%%%%%%%%%%%%%%%%%%%%%
\section{Preliminaries and Notation}
\label{sec:notation}

We collect here some notation that we will use throughout these notes. 
\tnote{If you introduce new notation in writing up a lecture, please consider
adding it here.} 


\paragraph{Basics.}
We most often denote sets by calligraphic, capital letters, such as $\calX$,
$\calY$, $\calZ$. A discrete probability distribution is a set $\Omega$, the
event space, together with a
function~$p\Colon\Omega\rightarrow[0,1]$ for which $\sum_{\omega\in\Omega}
p_\Omega(\omega) = 1$.
We write $p_\Omega$ when we need to disambiguate the event space, but
otherwise simply $p$ when $\Omega$ is clear from context. We will also,
following convention, often write $\Pr[\omega]$ instead of $p(\omega)$.
The support of $p$ is defined to be the set $\supp(p) = \{\omega \;|\; \omega
\in \Omega \land p(\omega) > 0\}$, 
i.e., the set of all points in $\Omega$ that have non-zero probability.  
As a slight abuse of notation, we can write 
$\Pr[\calS] = p(\calS) = \sum_{\omega\in\calS} p(\omega)$ for any set $\calS \subseteq \Omega$. 

A random variable $X$ is a map $Y\Colon\Omega\rightarrow\calX$ from some event
space $\Omega$ with associated probability distribution $p_\Omega$ over a  set
$\Omega$. So for some other set $\calS$ we let $\Pr[X\in\calS] = \Pr[\{\omega
\;|\; X(\omega) \in \calS\}]$ to be the probability that the random variable
takes on a value in $\calS$, over the probability distribution $p_\Omega$.

We use the language of probability as the foundation for formalizing
cryptographic algorithms, security, and more. Interestingly the probability
spaces involved get complicated quickly, and a common problem is that they end 
ambiguous. We will therefore rely on a crutch that has proved quite useful for
communicating, and reasoning about, probability distributions.


\paragraph{Pseudocode games.} We fix some pseudocode language to describe
security models, cryptographic algorithms, and more. Roughly we will follow the
notational tradition established by Bellare and Rogaway in the
2000s~\autocite{bellare2006security}, but using slightly different syntax/symantecs that are
based most closely on a treatment from~\autocite{ristenpart2011careful}.  Code-based
games are convenient for more precisely defining probability spaces, which are
what we use to model security and correctness goals, algorithms, and more.  

%We follow~\cite{BR06} 
%with some differences.
We will use procedures, variables, and typical
programming statements (such as operators, loops, procedure calls, etc.).
Typical statements are shown in \figref{fig:statements}.
We do not provide a formal specification of the programming language,
see~\cite[Appendix B]{bellare2006security} for an example of doing so. 
%Games include
%procedures, variables, and typical programming 
%statements (operators, loops, etc.). 
We will rely on some conventions to help make sense of games.
Types should be understable
from context. The names of syntactic objects (procedures, variables, etc.)
must be distinct.  Variables are implicitly 
initialized to default values:  integer variables are set
to 0, arrays are everywhere~$\bot$, etc. Here $\bot$ is a distinguished symbol
by tradition used to denote an error in the cryptographic literature.
By distinguished we mean that it assumed to not be used for any other reason,
not appearing in alphabets over which strings are taken, etc. We will often use $x \getsr \calX$ to denote that variable $x$ is assigned a value that is sampled uniformly at random from set $\calX$. 


\begin{wrapfigure}{r}{3in}
\gamesfontsize
\begin{tabular}{lp{2in}}
\toprule
$x \gets y$ & Assignment\\
$x \getsr \calX$ & Uniform sampling from a set\\
$z \getsr P(x,y)$ & Call a randomized procedure\\
$z \gets P(x,y)$ & Call a deterministic procedure\\
Ret $x$ & Return from a procedure\\
$z \gets x \concat y$ & String concatenation\\
$(x,y) \getparse{n} z$ & Parse string $z$ s.t.~$|x| = n$\\
\bottomrule
\end{tabular}
\caption{Some statements used in our games.}
\label{fig:statements}
\end{wrapfigure}


A \textit{procedure} is a sequence of statements together 
with zero or more variable inputs and zero or more 
outputs. %
%Variables are by default local, meaning they can only be 
%used within a single procedure, but they retain their state 
%between calls to the procedure. %
An \emph{unspecified procedure} is a procedure whose pseudocode, inputs, and
outputs are understood from context.  We will see some examples of
unspecified procedures, the most frequent in our security games being the 
\emph{adversary} which is often left unspecified. 
%
%
A call to a procedure requires providing it with inputs, running its sequence of
statements, and returning its output. We will interchangeably use the term call
and \emph{query} for procedure invocation.
A procedure~$P$ can itself query other
procedures. The set of procedures $Q_1,\ldots,Q_k$ called by a procedure are
statically fixed, and we require that there are no type mismatches in inputs and
outputs. 

Say that the code of~$P$ expects to be able to call~$k$ distinct procedures.
We will write $P^{Q_1,Q_2,\ldots,Q_k}$ to denote that these calls are
handled by $Q_1,Q_2,\ldots,Q_k$ and implicitly assume (for all $i\in[k]$) that there are no
syntactic mismatches between the calls that~$P$ makes to~$Q_i$ and the inputs
of~$Q_i$, as well as between the return values of~$Q_i$
and the return values expected by~$P$.   
%We stress that~$P$ does not
%call~$Q_i$ by name, but rather calls to a procedure that is
%instantiated by~$Q_i$.

We assume that all procedures eventually halt, regardless of randomness used, 
returning their outputs, at which point execution returns to the 
calling procedure.
Two procedures~$P_1$ and~$P_2$ are said to 
\emph{export the same interface} 
if their inputs and outputs have the same number and types. 

Variables will be local by default, meaning they can only be used within a
single procedure. Variables are static, meaning that they retain their 
state between calls to the procedure. It will be convenient to 
allow sharing of variables at times, for which we use a 
\textit{collection of procedures}. This is a set 
of one or more procedures whose variables have scope covering all of the
procedures. We will denote a collection of procedures using 
a common prefix ending with a period, so for example~$(P.x,P.y,\ldots)$,
Sometimes we will use the term interfaces for the specific prefixes of the a
collection of procedures~$P$. 

A \emph{main procedure} is a special procedure that takes no inputs and has some
output.  We will mark it by \main{} (though below
we'll see some syntactic sugar that provides greater brevity).  No procedure may
call~\main{}, it can access all the variables of other specified procedures
(though not unspecified procedures). 

A game consists of a main procedure together with a set of zero or more
specified procedures. We write $\G$ for a game. A game may also make use of
unspecified procedures (such as adversaries), which we enumerate as
superscripts, e.g., $\G^{P_1,P_2,\ldots,P_k}$. In most games used as security
definitions, one (or more) of the unspecified procedures will be called the
adversary, most often denoted by~$\advA$. For a given instantiation of the
unspecified procedures, one can run a game: execute its
statements starting with the designated \main{} procedure, and ultimately
outputing whatever \main{} returns.  

\paragraph{Run times and random variables.} Games can make random choices, due
to the supported statements for sampling according to a distribution. We can
associate to games a model of computation, which specifies how much running each
(type of) statement costs in terms of time, memory, or both.   A typical model
is to assign to each statement the same abstract unit cost, and the run time
then becomes the number of statements executed in the course of the game. 
When procedures are called, we attribute the unit cost of the call statement to
the caller and the remaining cost of executing the procedure's statements to the
callee. 

By default we will require that games terminate in some finite number of
steps~$\runtime$, and clarify explicitly when this does not hold. The number of
queries made by the main procedure, or any other procedure for that matter
including unspecified ones, is
therefore also upper bound by $\runtime$. We  may often limit the number of
queries made by an adversary to some maximum number $\numqueries \le \runtime$. 

Given these finiteness conditions, we similarly know that there is a finite
limit on the number of random samples made in a game. Since we restricted to 
random sampling from finite sets, we have that for any game $\G$ 
there is an event space~$\Omega_\G$ and an associated probability distribution
defining the output of the game $\G$. Given our restrictions, $\Omega_\G$ is a
set of possible values, the cross-product of all the random sampling procedures
within $\G$. We sometimes refer to $\Omega_\G$ as the \emph{coins} of the game.
For some fixed unspecified procedures
$P_1,\ldots,P_k$ we denote the event that executing the game
$\G^{P_1,\ldots,P_k}$ outputs a particular value $y$ by
``$\G^{P_1,\ldots,P_k}\Rightarrow y$'' and the associated probability over
$\Omega_\G$ is denoted $\Pr[\G^{P_1,\ldots,P_k}\Rightarrow y]$. When $y$ is clear
from context we will omit it, writing instead $\Pr[\G^{P_1,\ldots,P_k}]$.  
For example, we will often have games output a boolean and then
$y$ will most often be the value \true.

We can similarly associate to any variable within a game an event within
$\Omega_\G$. These can also be equivalently considered to be random variables on
domain $\Omega_G$. Our convention will be to overload notation, and define the
event that a variable $X$ in a game $\G$ takes on a certain value $y$ as simply
``$X = y$'' with associated probability $\Pr[X = y]$ over the coins of
$\Omega_\G$. 


\paragraph{Runnable games.}  We want to emphasize a point, which is that games
are by our conventions above runnable. That means that, once any unspecified
procedures are fixed, you could write a program in a conventional programming
language, and actually run the game on a real computer. Obviously like with all
abstract algorithms, the actual run times will vary, but assuming relative
efficiency the game will complete.  

It relatively frequently arises in proofs that one deviates from runnable games.
A common example is logic of the following form. Let $\advA$ be an adversary,
and consider a game $\G^\advA$. Remember we implicitly assume that $\advA$ is
compatible with $\G$, and we have that $\G^\advA$ is runnable.  Now consider the
set of all adversaries compatible with $\advA$, and let $\advA_{\max}$ be the
member of this set that maximizes $\Pr[\G^\advA \Rightarrow \true]$. But now
$\G^{\advA_{\max}}$ is no longer runnable, because $\advA_{\max}$ is not
concretely specified. While mathematically $\advA_{\max}$ is well-defined, there
is no clear way to write down its code, even given the code defining~$\advA$.
We will try whenever possible to avoid such arguments, as they have various
subtle implications that are, in general, not great for making 
clear claims about security. 





