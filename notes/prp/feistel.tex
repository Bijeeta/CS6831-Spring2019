%!TEX root = ../main.tex
%%%%%%%%%%%%%%%%%%%%%%%%%%%%%%%%%%%%%%%%%%%%%%%%%%%%%%%%%%%%%%%%%%%%%%%%%%%%%%%%
\subsection{Building PRPs from PRFs}
We've seen how it is hard to distinguish between a PRF and a PRP. \pfreadernote{This first sentence might be redundant given the next sentence}
In the previous section, we showed that it is hard to distinguish between a PRF and a PRP. \pfreadernote{Maybe worth referring back to \lemref{switching-lem}.}
However, as in the use case of block ciphers for length-preserving encryption and decryption, we specifically require a PRP for correctness purposes (in fact, we want an invertible PRP).
This is to ensure that a single ciphertext does not have ambiguous decryptions.
A natural question to ask is does the existence of a PRF imply the existence of a PRP, or more constructively, can we build a PRP given a PRF?

In this section, we will show that it is possible to build a PRP from a PRF.
We will examine one such construction called a Feistel network which is used in the construction of many early block ciphers, including DES (Data Encryption Standard), the first standardized block cipher.

\paragraph{Feistel networks}
A Feistel round transforms an arbitrary function into a permutation.
Define $\cipher:\braces*{0,1}^{2n}\rightarrow\braces*{0,1}^{2n}$ as the permutation constructed using one Feistel round with function $\prf:\braces*{0,1}^k\times\braces*{0,1}^n\rightarrow\braces*{0,1}^n$.
The function $\prf$ is known as the round function of the Feistel network.
\begin{wrapfigure}{r}{2in}
\center
\begin{tikzpicture}
    \node(r0) at ($(0,0)$)  {$R_0$};
    \node (l0) [left of = r0, node distance=3.0cm] {$L_0$};
    \node[draw,thick,minimum width=0.75cm] (\prf) [below of = r0, node distance=1.1cm]  {$\prf_{\prfkey}$};
    \node (xor) [XOR, below of = \prf, node distance = 0.75cm, scale=1.0] {};
    \node (junction0) [below of = r0, node distance = 0.5cm] {};
    \node (junction1) [below of = l0, node distance = 0.75cm] {};
    \node (junction2) [below of = l0, node distance = 1.5cm] {};
    \node (junction3) [left of = xor, node distance = 0.5cm] {};
    \path[->]
      (r0) edge[thick] node {} (\prf)
      (\prf) edge[thick] node {} (xor)
      (junction3.center) edge[thick] node {} (xor);
    \path[-]
      (l0.south) edge[thick] node {} (junction1.center)
      (junction0.center) edge[thick] node {} (junction2.center)
      (junction1.center) edge[thick] node {} (junction3.center);
    \node (r1) [below of = xor, node distance=0.75cm] {$R_1$};
    \node (l1) [left of = r1, node distance=3.0cm] {$L_1$};
    \path[->]
      (xor) edge[thick] node {} (r1)
      (junction2.center) edge[thick] node {} (l1);
\end{tikzpicture}
\caption{A single round of a Feistel network.}
\label{fig:feistel-1}
\end{wrapfigure}
This construction is depicted in Figure~\ref{fig:feistel-1} where $2n$ length inputs and outputs are split into left and right parts of length $n$ notated $L$ and $R$, respectively.
Function $\prf$ is notated with fixed key $\prfkey$ as $\prf_{\prfkey}:\braces*{0,1}^n\rightarrow\braces*{0,1}^n$.
The Feistel round is then
\begin{align*}
L_1 &\gets R_0\\
R_1 &\gets L_0 \oplus \prf_{\prfkey}(R_0)\,.
\end{align*}


\begin{example}[1-round Feistel network is a permutation]
  \pfreadernote{Should examples be in sub-sections? i.e., example 3.1, 3.2.} First, let us see why the Feistel construction builds a permutation from an arbitrary function.
  Notice that as long as the round function $\prf_{\prfkey}$ is defined on all inputs from domain $\braces*{0,1}^n$ then $\cipher$ is defined on all inputs from domain $\braces*{0,1}^{2n}$.
  Next, we show that any output of $\cipher$, $L_1\concat R_1$, corresponds to a unique input $L_0\concat R_0$.
  Even though $\prf_{\prfkey}$ is not assumed to be invertible, since the input to $\prf_{\prfkey}$, $R_0$, is passed directly to the output as $L_1$, we can invert as follows
  \begin{align*}
    L_0 &\gets R_1\oplus \prf_{\prfkey}(L_1)\\
    R_0 &\gets L_1\,.
  \end{align*}
  These two properties together mean $\cipher$ is a complete permutation on $\braces*{0,1}^{2n}$.
\end{example}

\begin{example}[1-round Feistel network is not a PRP]
  Next, we show that a 1-round Feistel network $\cipher$ is not a PRP.
  Consider the following PRP adversary $\advA$ for domain $\braces*{0,1}^{2n}$:
  \begin{center}
    \fpage{.2}{
    \underline{\textbf{adversary }$\advA^{\Fn}$}\\[2pt]
    $(L_1,R_1)\gets\Fn(0^{2n})$\\
    Return $L_1 = 0^{n}$
    }
  \end{center}

  In $\PRP1$, where the oracle $\Fn$ returns $\cipher_{\prfkey}(0^{2n})$, $\advA$ returns 1 with probability 1.
  In $\PRP0$, where oracle $\Fn$ returns a random value, the probability $\advA$ returns 1 is the probability that the first $n$ bits of the random output are 0.
  Thus,
\begin{align*}
\AdvPRP{\cipher}{\advA} &= \absv*{\Prob{\PRP1_\cipher^\advA\Rightarrow 1} - \Prob{\PRP0_\cipher^\advA\Rightarrow1} }\\
&\geq 1 - \frac{1}{2^n}\,.   
\end{align*}
\end{example}
\pfreadernote{I think the above is just equals?}
We see that it is trivial to distinguish a 1-round Feistel network from a random permutation, since the second half of the input is mapped directly to the first half of the output.
Does the randomness of the permutation improve by performing numerous Feistel rounds?
One way to construct a multi-round Feistel network is by feeding the outputs of the previous round as inputs to the next round, where each round is keyed with a different unique key.
To use only a single key, we could consider an alternate construction, where the round function $\prf$ in each Feistel round has a different domain, $\prf:\braces{0,1}^{k}\times\braces{0,1}^{2n}\rightarrow\braces{0,1}^n$.
As before, the round function will take in the second half of the input, but it will also take in an $n$-bit round counter.
To stack Feistel rounds into a multi-round Feistel network, we simply ensure that each round uses a different round counter.
Figure~\ref{fig:feistel-3} depicts a 3-round Feistel network constructed in this manner.

\begin{wrapfigure}{r}{2.5in}
\center
\begin{tikzpicture}
    \node(r0) at ($(0,0)$)  {$R_0$};
    \node (l0) [left of = r0, node distance=3.0cm] {$L_0$};
    \def\lastz{0}
    \foreach \z[remember=\z as \lastz] in {1, 2,...,3} {
      \node[draw,thick,minimum width=0.75cm] (\prf\z) [below of = r\lastz, node distance=1.1cm]  {$\prf_{\prfkey}$};
      \node (ctr\z) [right of = \prf\z, node distance=1.25cm] {$\bm{\langle}\z\bm{\rangle}$};
      \node (xor\z) [XOR, below of = \prf\z, node distance = 0.75cm, scale=1.0] {};
      \node (junction0\z) [below of = r\lastz, node distance = 0.5cm] {};
      \node (junction1\z) [below of = l\lastz, node distance = 0.75cm] {};
      \node (junction2\z) [below of = l\lastz, node distance = 1.5cm] {};
      \node (junction3\z) [left of = xor\z, node distance = 0.5cm] {};
      \node (r\z) [below of = xor\z, node distance=0.75cm] {$R_\z$};
      \node (l\z) [left of = r\z, node distance=3.0cm] {$L_\z$};
      \path[->]
        (r\lastz) edge[thick] node {} (\prf\z)
        (\prf\z) edge[thick] node {} (xor\z)
        (junction3\z.center) edge[thick] node {} (xor\z)
        (xor\z) edge[thick] node {} (r\z)
        (junction2\z.center) edge[thick] node {} (l\z)
        (ctr\z) edge[thick] node {} (\prf\z.east);
        \path[-]
        (l\lastz.south) edge[thick] node {} (junction1\z.center)
        (junction0\z.center) edge[thick] node {} (junction2\z.center)
        (junction1\z.center) edge[thick] node {} (junction3\z.center);
    }
\end{tikzpicture}
\caption{A 3-round Feistel network.}
\label{fig:feistel-3}
\end{wrapfigure}

\begin{example} A 2-round Feistel network is not a PRP.
  \scribenote{Add as homework problem}
\end{example}
