%!TEX root = ../main.tex
%%%%%%%%%%%%%%%%%%%%%%%%%%%%%%%%%%%%%%%%%%%%%%%%%%%%%%%%%%%%%%%%%%%%%%%%%%%%%%%%
\paragraph{3-round Feistel network is a PRP}
Next, we show that given PRF security of the underlying round function, a 3-round Feistel network is a PRP.
Intuitively, the proof will follow the intuition that 3 rounds of Feistel allows the round function to properly add randomness to both $L_3 = R_2$ and $R_3$.
In particular, we want to show that the output of the $\prf_{\prfkey}(\langle 2 \rangle\concat R_1)$ and $\prf_{\prfkey}(\langle 3 \rangle\concat R_2)$ cannot be effectively controlled by adversary $\advA$, i.e., $\advA$ cannot create collisions on $R_1$ and $R_2$.

\begin{theorem}[Luby-Rackoff]\label{thm:luby-rackoff}
Let $\Feistel$ be the 3-round Feistel cipher using round function
$\prf\Colon\bits^k\times\bits^{2n}\rightarrow \bits^n$. For any
$\PRP_\cipher$-adversary $\advA$ making at most $q$ queries
we give a $\PRF_{\prf}$-adversary $\advB$ making at most $3q$ queries such that
\bnm
  \AdvPRP{\Feistel}{\advA} \le \AdvPRF{\prf}{\advB} + \frac{2q^2}{2^n} +
  \frac{q^2}{2^{2n}} \;.
\enm
\end{theorem}
\scribenote{How are we doing cites? (Luby-Rackoff)}

\begin{proof}
Without loss of generality, we consider an adversary $\advA$ that only makes unique queries to oracle $\Fn$.
\scribenote{Be more formal about why this okay?} \pfreadernote{Looking at the Boneh-Shoup explanation for this, I think it would be better to be more clear on why this is ok}
We bound the advantage of $\PRP$ adversary $\advA$ by bounding its advantage in each game in a series of game hops.
Pseudocode for these games is given in Figure~\ref{fig:games-luby-rackoff}.

Game $\G0$ is constructed exactly as $\PRP1^{\advA}_{\Feistel}$; the oracle pseudocode expands out the computation of the 3-round Feistel network $\Feistel$:
\bnm
\Prob{\PRP1_\Feistel^\advA\Rightarrow 1} = \Prob{\G0\Rightarrow 1}.
\enm
Game $\G1$ replaces the round function $\prf$ with a random function $\randfn$ mapping from $\braces{0,1}^{2n}\rightarrow\braces{0,1}^n$.
\scribenote{Should PRF security game be defined with respect to different domain and range space, $m$ and $n$?}
We bound the ability to distinguish between $\G0$ and $\G1$ by the PRF security of $\prf$.
Consider the following PRF adversary $\advB$ which runs $\advA$ with a simulated oracle $\FnSim$.

\begin{center}
\fpage{.22}{
\underline{$\advB^\Fn$}\\[2pt]
$K \getsr \bits^k$\\
$b' \getsr \advA^\FnSim$\\
Return $b'$\medskip

\underline{$\FnSim(\msg)$}\\
$L_1 \gets R_0$\\
$R_1 \gets L_0 \oplus \Fn(\langle 1\rangle \concat R_0)$\\
$L_2 \gets R_1$\\
$R_2 \gets L_1 \oplus \Fn(\langle 2\rangle \concat R_1)$\\
$L_3 \gets R_2$\\
$R_3 \gets L_2 \oplus \Fn(\langle 3\rangle \concat R_2)$\\
Return $L_3 \concat R_3$
}
\end{center}
\pfreadernote{Should this game be a figure?} \pfreadernote{Julia: since it's just an adversary, it's probably ok?}
Adversary $\advB$ simulates $\FnSim$ by running the 3-round Feistel network but replacing the round function with a call to its own oracle $\Fn$.
In \PRF0, where $\advB$'s oracle $\Fn$ acts as the round function $\prf$, $\advB$ runs exactly $\G0$.
In \PRF1, where $\Fn$ acts as a random function $\rho$, $\advB$ runs exactly $\G1$.
Thus, we have
\begin{align*}
\AdvPRF{\prf}{\advB} &= \absv*{\Prob{\PRF1_\prf^\advB\Rightarrow 1} - \Prob{\PRF0_\prf^\advB\Rightarrow1}}\\
&= \absv*{\Prob{\G1\Rightarrow 1} - \Prob{\G0\Rightarrow1}}.\\
\end{align*}

\begin{figure}[t]
\hfpagessss{.20}{.20}{.20}{.25}{
\underline{$\G0$}\\[2pt]
$K \getsr \bits^k$\\
$b' \getsr \advA^\Fn$\\
Return $b'$\medskip

\underline{$\Fn(\msg)$}\\
$(L_0, R_0) \gets \msg$\\
$L_1 \gets R_0$\\
$R_1 \gets L_0 \oplus F_K(\langle 1\rangle \concat R_0)$\\
$L_2 \gets R_1$\\
$R_2 \gets L_1 \oplus F_K(\langle 2 \rangle \concat R_1)$\\
$L_3 \gets R_2$\\
$R_3 \gets L_2 \oplus F_K(\langle 3 \rangle \concat R_2)$\\
Return $L_3 \concat R_3$
}{
\underline{$\G1$}\\[2pt]
$\rho \getsr \Func(2n,n)$\\
$b' \getsr \advA^\Fn$\\
Return $b'$\medskip

\underline{$\Fn(\msg)$}\\
$(L_0, R_0) \gets \msg$\\
$L_1 \gets R_0$\\
$R_1 \gets L_0 \oplus \rho(\langle 1\rangle \concat R_0)$\\
$L_2 \gets R_1$\\
$R_2 \gets L_1 \oplus \rho(\langle 2 \rangle \concat R_1)$\\
$L_3 \gets R_2$\\
$R_3 \gets L_2 \oplus \rho(\langle 3 \rangle \concat R_2)$\\
Return $L_3 \concat R_3$
}{
\underline{$\fbox{\G2}$\;\;\;\G3}\\[2pt]
$b' \getsr \advA^\Fn$\\
Return $b'$\medskip

\underline{$\Fn(\msg)$}\\
$(L_0, R_0) \gets \msg$\\
$L_1 \gets R_0$\\
If $\TabF[\langle1\rangle\concat R_0] = \bot$ then\\
\ind $\TabF[\langle1\rangle\concat R_0] \getsr \bits^n$\\
$X_1 \gets \TabF[\langle1\rangle\concat R_0]$\\
$R_1 \gets L_0 \oplus X_1$\\
$L_2 \gets R_1$\\
$X_2 \getsr \bits^n$\\
If $\TabF[\langle2\rangle\concat R_1] \ne \bot$ then\\
\ind $\badtrue$\\
\ind \fbox{$X_2 \gets \TabF[\langle2\rangle\concat R_1]$}\\
$\TabF[\langle2\rangle\concat R_1] \gets X_2$\\
$R_2 \gets L_1 \oplus X_2$\\
$L_3 \gets R_2$\\
$X_3 \getsr \bits^n$\\
If $\TabF[\langle3\rangle\concat R_2] \ne \bot$ then\\
\ind $\badtrue$\\
\ind \fbox{$X_3 \gets \TabF[\langle3\rangle\concat R_2]$}\\
$\TabF[\langle3\rangle\concat R_2] \gets X_3$\\
$R_3 \gets L_2 \oplus X_3$\\
Return $L_3 \concat R_3$
}{
\underline{$\G3$\;\;\;$\fbox{\G4}$}\\[2pt]
$b' \getsr \advA^\Fn$\\
Return $b'$\medskip

\underline{$\Fn(\msg)$}\\
$(L_0, R_0) \gets \msg$\\
$L_1 \gets R_0$\\
If $\TabF[\langle1\rangle\concat R_0] = \bot$ then\\
\ind $\TabF[\langle1\rangle\concat R_0] \getsr \bits^n$\\
$X_1 \gets \TabF[\langle1\rangle\concat R_0]$\\
$R_1 \gets L_0 \oplus X_1$\\
$L_2 \gets R_1$\\
$X_2 \getsr \bits^n$\\
If $\TabF[\langle2\rangle\concat R_1] \ne \bot$ then\\
\ind $\badtrue$\\
$\TabF[\langle2\rangle\concat R_1] \gets X_2$\\
$R_2 \gets L_1 \oplus X_2$;\;\;\fbox{$R_2 \getsr \bits^n$}\\
$L_3 \gets R_2$\\
$X_3 \getsr \bits^n$\\
If $\TabF[\langle3\rangle\concat R_2] \ne \bot$ then\\
\ind $\badtrue$\\
$\TabF[\langle3\rangle\concat R_2] \gets X_3$\\
$R_3 \gets L_2 \oplus X_3;$\;\;\fbox{$R_3 \getsr \bits^n$}\\
Return $L_3 \concat R_3$
}
\caption{Games for proof of 3-round Feistel network as PRP (Theorem~\ref{thm:luby-rackoff})}
\label{fig:games-luby-rackoff}
\end{figure}

In games $\G2$ and $\G3$, we use a similar trick to the one we used in our proof of the PRF-PRP switching lemma.
Game $\G2$ replaces the random function $\rho$ with a lazy evaluation of a random function using a table $\TabF$.
When the random function is queried on an input, the value in $\TabF$ keyed by the input is returned.
If there is no such value, meaning the input has not been previously queried, a new random value is generated, returned, and stored in $\TabF$ for future queries.
Thus, lazily building out $\TabF$ is equivalent to using random function $\rho$:
\bnm
\Prob{\G1\Rightarrow 1} = \Prob{\G2\Rightarrow 1}.
\enm

Game $\G3$ generates fresh random values on every input to the second and third round functions.
This is in contrast to $\G2$ where a fresh random value is only returned for inputs that have never been seen.
Thus, the difference between $\G2$ and $\G3$ occur when repeat inputs are used with the second or third round functions, which corresponds to repeat values of $R_1$ and $R_2$, respectively.
Intuitively, since we are only considering adversaries $\advA$ that make unique queries to $\Fn$, finding repeat values of $R_1$ and $R_2$ is hard because they both include randomness from the random function.
The pseudocode in Figure~\ref{fig:games-luby-rackoff} captures the event of a repeat query by a setting a $\bad$ flag.
The only difference between $\G2$ and $\G3$ occur after the $\bad$ flag is set; $\G2$ returns the consistent value from the look-up table $\TabF$, while $\G3$ returns a fresh random value.
Then by the fundamental lemma of game-playing,
\bnm
\absv*{\Prob{\G3\Rightarrow 1} - \Prob{\G2\Rightarrow1}} \le \Prob{\G3 \setsbad}.
\enm

Game $\G4$ is the same as $\G3$ except $R_2$ and $R_3$ are set to fresh random values.
In $\G3$, we have $R_2\gets L_1\oplus X_2$ and $R_3\gets L_2\oplus X_3$ for fresh random values $X_2,X_3$.
Thus, the probability space of $R_2$ and $R_3$ are the same and $\G4$ is identical to $\G3$:
\bnm
\Prob{\G3\Rightarrow 1} = \Prob{\G4\Rightarrow 1}.
\enm
Importantly, this also implies
\bnm
\Prob{\G3 \setsbad} = \Prob{\G4 \setsbad}.
\enm

Notice that $\G4$ simply returns a fresh random string of length $2n$ on every query to oracle $\Fn$.
Since we assume $\advA$ only makes unique queries to $\Fn$, $\G4$ is simulating a perfect random function, and thus is equivalent to the \PRF0 game:
\bnm
\Prob{\G4\Rightarrow 1} = \Prob{\PRF0^{\advA}_{\Feistel}\Rightarrow 1}.
\enm

And by the PRF-PRP switching lemma,
	\bnm
	\absv*{\Prob{\PRF0_{\Feistel}^{\advA}\Rightarrow 1} - \Prob{\PRP0_{\Feistel}^{\advA}\Rightarrow1}} \le \frac{q^2}{2^{2n}}  \;.
	\enm

Lastly, we can bound the probability that $\G4\setsbad$.
First, consider the probability $\G4\setsbad$ on query $i$; call this event $W_i$.
This event can be considered as two separate events depending on where the $\bad$ flag is set:
denote the event that $R_1$ collides as $W_i^1$ and the event that $R_2$ collides as $W_i^2$.

It is simple to bound the probability of $W_i^2$.
The probability of $R_2 = L_1 \xor X_2$ colliding with a previous $R_2$ on query $i$ is bounded by $q/2^n$ since $X_2$ is a fresh random string. \pfreadernote{$\G4$ just samples $R_2$ randomly, not $R_2 = L_1 \xor X_2$ so it might be good to refresh the reader's memory that we're using Pr[G3 sets bad] = Pr[G4 sets bad] here. Also maybe mention union bound for how you upper bounded the probability?}

Bounding the probability of $W_i^1$ is more nuanced.
At first glance, it appears that the logic should follow symmetrically to that from above where $R_1 = L_0\xor X_1$ for random $X_1$.
To bound the probability of collision to $q/2^n$, we must argue that $X_1$ is independent of the inputs to $\Fn$.
In the previous case, $X_2$ is freshly sampled on every query to $\Fn$ so independence is trivially satisfied.
This is not the case for $X_1$.
However, we can argue that $\advA$'s queries are dependent only on $\advA$'s random coins and the previous outputs of $\Fn$.
Since in $\G4$ the outputs of $\Fn$ are independently drawn random samples, we have that the inputs of $\Fn$ are independent of $X_1$.
\scribenote{May need to argue this more formally?}

Thus, by two applications of the union bound, we have
\pfreadernote{specify that initial value for sum is i=1 below}
\begin{align*}
  \Prob{\G4 \setsbad} &\le \sum_i^q \Prob{W_i}\\
  &\le \sum_i^q \Prob{W_i^1} + \Prob{W_i^2}\\
  &\le \sum_i^q \frac{q}{2^n} + \frac{q}{2^n}\\
  &= \frac{2q^2}{2^n}.
\end{align*}

Finally, to put it all together,

\begin{align*}
\AdvPRP{\Feistel}{\advA}
    &=\left|\Prob{\PRP1^\advA_\Feistel} - \Prob{\PRP0^\advA_\Feistel}\right|\\
    &= \left|\Prob{\G0} - \Prob{\PRP0^\advA_\cipher}\right|\\
    &\le \left|\Prob{\G1} + \AdvPRF{F}{\advB} - \Prob{\PRP0^\advA_\cipher}\right|\\
    &=   \left|\Prob{\G2} + \AdvPRF{F}{\advB} - \Prob{\PRP0^\advA_\cipher}\right|\\
    &\le \left|\Prob{\G3} + \Prob{\G3\setsbad} + \AdvPRF{F}{\advB} - \Prob{\PRP0^\advA_\cipher}\right|\\
    &= \left|\Prob{\G4} + \Prob{\G4\setsbad} + \AdvPRF{F}{\advB} - \Prob{\PRP0^\advA_\cipher}\right|\\
    &\le \left|\Prob{\G4} + \frac{2q^2}{2^n} + \AdvPRF{F}{\advB} - \Prob{\PRP0^\advA_\cipher}\right|\\
    &\le \left|\Prob{\PRP0^\advA_\cipher} + \frac{q^2}{2^{2n}} + \frac{2q^2}{2^n} + \AdvPRF{F}{\advB} - \Prob{\PRP0^\advA_\cipher}\right|\\
    &= \frac{q^2}{2^{2n}} + \frac{2q^2}{2^n} + \AdvPRF{F}{\advB}\\
\end{align*}

\end{proof}

\scribenote{Additional Exercise Ideas}
\begin{itemize}
  \item 2-round Feistel is not a PRP
  \item Show security of 3-round Feistel with 3 different keys
  \item What happens when same key is used across Feistel rounds? (I don't know)
  \item Show 3-round Feistel is not Strong PRP (Strong PRP gets access to both forward and inverse oracle) \pfreadernote{this would require providing strong PRP game}
  \item Show 4-round Feistel is Strong PRP
\end{itemize}
