%!TEX root = ../main.tex
%%%%%%%%%%%%%%%%%%%%%%%%%%%%%%%%%%%%%%%%%%%%%%%%%%%%%%%%%%%%%%%%%%%%%%%%%%%%%%%%
\paragraph{Connection to card shuffling algorithms.}
It turns out that building PRP block ciphers, perhaps unsurprisingly, is highly related to literature in designing ``good'' card shuffling algorithms.
Shuffling a deck of cards can be thought of as a permutation in which the position of each card pre-shuffle is mapped to its position post-shuffle.
A ``good'' shuffle desires that all possible deck permutations are close-to equally likely, similar to the goals of a PRP.
We will examine the connection of the Thorp shuffle~\cite{Thorp73} to the Feistel construction.

The Thorp shuffle, depicted in~\figref{fig:unbalanced-card-shuffle}, works as follows.
To shuffle a deck of $N$ cards, first cut the deck into two equal sized piles.
Drop the bottom card from each of the two piles.
Flip a coin to determine the order in which to place the two dropped cards in the new shuffled deck.
Continue for a total of $N/2$ draws in which the card at position $i$ is drawn with the card at position $i+N/2$ and a coin is flipped to determine their position between $2i$ and $2i+1$ in the shuffled deck.
Together, this is called one pass of the Thorp shuffle.

The connection of the Thorp shuffle to cryptography and block cipher design was first noticed by Naor~\cite{NaorReingold99}.
Consider the maximally unbalanced Feistel construction given in~\figref{fig:unbalanced-card-shuffle}.
An unbalanced Feistel construction splits its inputs into unequal sized bit strings and the round function compresses the larger string to the size of the smaller string.
In this example of a maximally unbalanced construction, the input is split into two strings of length $1$ bit and $n-1$ bits.
Observe that if the round function of the Feistel construction was a pseudorandom function, it would perform exactly one pass of the Thorp shuffle.
Consider the positions of the cards as bit strings of length $n = \lg N$.
With position $i$ in the first half of the deck, the output of the pseudorandom function on the bit representation of $i$ acts as the coin flip for determining card $i$ and card $i + N/2$ placement in position $2i$ or $2i+1$.

This connection can be generalized to the balanced Feistel construction by designing a ``balanced'' Thorp shuffle.
This shuffle, depicted in~\figref{fig:balanced-card-shuffle}, splits the deck into $\sqrt{N}$ equal piles of size $\sqrt{N}$.
Similarly, one card from each pile is drawn and together are placed in a random order.
This corresponds to the balanced Feistel construction splitting the inputs into strings of length $n/2$.
The output of the round function determines the position of the card when placing the $\sqrt{N}$ draws in random order.

With this connection, we can ask does every card shuffling algorithm correspond to some block cipher construction?
Unfortunately not.
Naor~\cite{NaorReingold99} observes that a shuffling algorithm corresponds to a block cipher if and only if it is \emph{oblivious}, meaning that one can trace the route of any given card in the deck without attending to the remaining cards in the deck.
It is easy to see the Thorp shuffle is oblivious as the next position of a card is only dependent on its current position and a random coin flip.
The security of the Thorp shuffle as a block cipher was shown by Morris et al.~\cite{MRS09}.



\tikzset{card/.append style={draw,minimum width=0.75cm,minimum height=1.0cm,rounded corners=0.25cm}}
\newcommand{\thorpcarddist}{0.3cm}
\newcommand{\thorpcarddisthalf}{0.15cm}
\newcommand{\thorpcarddisthalfplus}{0.502cm}
\newcommand{\thorplabeldist}{0.05cm}
\newcommand{\thorpcarddivtop}{0.75cm}
\newcommand{\thorpcarddivbot}{0.45cm}
\newcommand{\thorpshuffledist}{1.4cm}
\newcommand{\thorpshuffledisthalf}{0.7cm}
\newcommand{\thorpcarddistshort}{0.15cm}

\begin{figure}[t]
  \centering
\begin{tikzpicture}[every node/.style={inner sep=0pt}]
\node[card] (c0) at ($(0,0)$) {A\ding{171}};
\node[card] (c1) [right=\thorpcarddist of c0] {2\ding{171}};
\node[card] (c2) [right=\thorpcarddist of c1] {3\ding{171}};
\node[card] (c3) [right=\thorpcarddist of c2] {4\ding{171}};
\node[card] (c4) [right=\thorpcarddist of c3] {5\ding{171}};
\node[card] (c5) [right=\thorpcarddist of c4] {6\ding{171}};
\node[card] (c6) [right=\thorpcarddist of c5] {7\ding{171}};
\node[card] (c7) [right=\thorpcarddist of c6] {8\ding{171}};
\node (i0) [above=\thorplabeldist of c0] {\texttt{000}};
\node (i1) [above=\thorplabeldist of c1] {\texttt{001}};
\node (i2) [above=\thorplabeldist of c2] {\texttt{010}};
\node (i3) [above=\thorplabeldist of c3] {\texttt{011}};
\node (i4) [above=\thorplabeldist of c4] {\texttt{100}};
\node (i5) [above=\thorplabeldist of c5] {\texttt{101}};
\node (i6) [above=\thorplabeldist of c6] {\texttt{110}};
\node (i7) [above=\thorplabeldist of c7] {\texttt{111}};

\node[card] (s0) [below=\thorpshuffledist of c0] {A\ding{171}};
\node[card] (s1) [right=\thorpcarddist of s0] {5\ding{171}};
\node[card] (s2) [right=\thorpcarddist of s1] {6\ding{171}};
\node[card] (s3) [right=\thorpcarddist of s2] {2\ding{171}};
\node[card] (s4) [right=\thorpcarddist of s3] {7\ding{171}};
\node[card] (s5) [right=\thorpcarddist of s4] {3\ding{171}};
\node[card] (s6) [right=\thorpcarddist of s5] {8\ding{171}};
\node[card] (s7) [right=\thorpcarddist of s6] {4\ding{171}};
\node (o0) [below=\thorplabeldist of s0] {\texttt{000}};
\node (o1) [below=\thorplabeldist of s1] {\texttt{010}};
\node (o2) [below=\thorplabeldist of s2] {\texttt{101}};
\node (o3) [below=\thorplabeldist of s3] {\texttt{001}};
\node (o4) [below=\thorplabeldist of s4] {\texttt{110}};
\node (o5) [below=\thorplabeldist of s5] {\texttt{010}};
\node (o6) [below=\thorplabeldist of s6] {\texttt{111}};
\node (o7) [below=\thorplabeldist of s7] {\texttt{011}};

\foreach \z in {0, 1,...,6} {
  \node (jcm\z) [right=\thorpcarddisthalf of c\z] {};
  \node (jct\z) [above=\thorpcarddivtop of jcm\z] {};
  \node (jcb\z) [below=\thorpcarddivbot of jcm\z] {};

  \node (jsm\z) [right=\thorpcarddisthalf of s\z] {};
  \node (jst\z) [above=\thorpcarddivbot of jsm\z] {};
  \node (jsb\z) [below=\thorpcarddivtop of jsm\z] {};

  \node (jm\z) [below=\thorpshuffledisthalf of c\z] {};
  \node (jmm\z) [right=\thorpcarddisthalfplus of jm\z] {};
}

\node[draw,fill,circle,minimum width=0.1cm] (shuf1) [right=\thorpcarddisthalfplus of jm0.center] {};
\node[draw,fill,circle,minimum width=0.1cm] (shuf2) [right=\thorpcarddisthalfplus of jm2.center] {};
\node[draw,fill,circle,minimum width=0.1cm] (shuf3) [right=\thorpcarddisthalfplus of jm4.center] {};
\node[draw,fill,circle,minimum width=0.1cm] (shuf4) [right=\thorpcarddisthalfplus of jm6.center] {};

\path[dashed,-]
  (jct3.center) edge[thick] node {} (jcb3.center)
  (jst1.center) edge[thick] node {} (jsb1.center)
  (jst3.center) edge[thick] node {} (jsb3.center)
  (jst5.center) edge[thick] node {} (jsb5.center);
\path[-]
  (c0.south) edge node {} (shuf1.center)
  (c4.south) edge node {} (shuf1.center)
  (c1.south) edge node {} (shuf2.center)
  (c5.south) edge node {} (shuf2.center)
  (c2.south) edge node {} (shuf3.center)
  (c6.south) edge node {} (shuf3.center)
  (c3.south) edge node {} (shuf4.center)
  (c7.south) edge node {} (shuf4.center);
\path[->]
  (shuf1.center) edge node {} (s0)
  (shuf1.center) edge node {} (s1)
  (shuf2.center) edge node {} (s2)
  (shuf2.center) edge node {} (s3)
  (shuf3.center) edge node {} (s4)
  (shuf3.center) edge node {} (s5)
  (shuf4.center) edge node {} (s6)
  (shuf4.center) edge node {} (s7);
\end{tikzpicture}
\hspace{4em}
\begin{tikzpicture}
    \node[draw,thick,minimum width=3.5cm] (r0) at ($(0,0)$)  {$R_0$};
    \node[draw,thick,minimum width=0.5cm] (l0) [left of = r0, node distance=2.5cm] {$L_0$};
    \node (r0label) [above=0.1cm of r0] {$n-1$ bits};
    \node (l0label) [above=0.1cm of l0] {$1$ bit};
    \node[draw,thick,minimum width=0.75cm] (\prf) [below=0.75cm of r0.south east, anchor=north east]  {$\prf_{\prfkey}$};
    \node (xor) [XOR, below of = \prf, node distance = 0.75cm, scale=1.0] {};
    \node[minimum width=0.75cm] (junction0) [below=0.25cm of r0.south east, anchor=north east] {};
    \node (junction1) [below of = l0, node distance = 0.75cm] {};
    \node (junction2) [below of = l0, node distance = 2.0cm] {};
    \node (junction3) [left of = xor, node distance = 0.5cm] {};
    \path[->]
      (r0.347) edge[thick] node {} (\prf)
      (\prf) edge[thick] node {} (xor)
      (junction3.center) edge[thick] node {} (xor);
    \path[-]
      (l0.south) edge[thick] node {} (junction1.center)
      (junction0.center) edge[thick] node {} (junction2.center)
      (junction1.center) edge[thick] node {} (junction3.center);
    \node[draw,thick,minimum width=0.5cm] (r1) [below of = xor, node distance=0.75cm] {$R_1$};
    \node[draw,thick,minimum width=3.5cm] (l1) [left of = r1, node distance=2.5cm] {$L_1$};
    \path[->]
      (xor) edge[thick] node {} (r1)
      (junction2.center) edge[thick] node {} (l1.167);
\end{tikzpicture}
\caption{Thorp shuffle (left).
The deck is split in half. In turn, two cards, one from each half deck, are drawn and randomly placed.
The resulting permutation matches that of a maximally unbalanced Feistel network (right), where the randomness of the card placements corresponds to a possible Feistel round function.
}
\label{fig:unbalanced-card-shuffle}
\end{figure}

\begin{figure}
  \centering
\makebox[\textwidth][c]{\resizebox{\textwidth}{!}{
\begin{tikzpicture}[every node/.style={inner sep=0pt}]
\node[card] (s1) at ($(0,0)$) {A\ding{171}};
\node[card] (s2) [right=\thorpcarddistshort of s1] {2\ding{171}};
\node[card] (s3) [right=\thorpcarddistshort of s2] {3\ding{171}};
\node[card] (s4) [right=\thorpcarddistshort of s3] {4\ding{171}};
\node[card] (h1) [right=\thorpcarddist of s4] {A\ding{170}};
\node[card] (h2) [right=\thorpcarddistshort of h1] {2\ding{170}};
\node[card] (h3) [right=\thorpcarddistshort of h2] {3\ding{170}};
\node[card] (h4) [right=\thorpcarddistshort of h3] {4\ding{170}};
\node[card] (d1) [right=\thorpcarddist of h4] {A\ding{169}};
\node[card] (d2) [right=\thorpcarddistshort of d1] {2\ding{169}};
\node[card] (d3) [right=\thorpcarddistshort of d2] {3\ding{169}};
\node[card] (d4) [right=\thorpcarddistshort of d3] {4\ding{169}};
\node[card] (c1) [right=\thorpcarddist of d4] {A\ding{168}};
\node[card] (c2) [right=\thorpcarddistshort of c1] {2\ding{168}};
\node[card] (c3) [right=\thorpcarddistshort of c2] {3\ding{168}};
\node[card] (c4) [right=\thorpcarddistshort of c3] {4\ding{168}};

\foreach \z in {s, h, d} {
  \node (jm\z) [right=\thorpcarddisthalf of \z4] {};
  \node (jt\z) [above=\thorpcarddivbot of jm\z] {};
  \node (jb\z) [below=\thorpcarddivbot of jm\z] {};
}
\path[dashed,-]
  (jts.center) edge[thick] node {} (jbs.center)
  (jth.center) edge[thick] node {} (jbh.center)
  (jtd.center) edge[thick] node {} (jbd.center);

\node[card] (A1) [below=\thorpshuffledist of s1] {A\ding{170}};
\node[card] (A2) [right=\thorpcarddistshort of A1] {A\ding{168}};
\node[card] (A3) [right=\thorpcarddistshort of A2] {A\ding{171}};
\node[card] (A4) [right=\thorpcarddistshort of A3] {A\ding{169}};
\node[card] (B1) [right=\thorpcarddist of A4] {2\ding{171}};
\node[card] (B2) [right=\thorpcarddistshort of B1] {2\ding{170}};
\node[card] (B3) [right=\thorpcarddistshort of B2] {2\ding{168}};
\node[card] (B4) [right=\thorpcarddistshort of B3] {2\ding{169}};
\node[card] (C1) [right=\thorpcarddist of B4] {3\ding{168}};
\node[card] (C2) [right=\thorpcarddistshort of C1] {3\ding{171}};
\node[card] (C3) [right=\thorpcarddistshort of C2] {3\ding{169}};
\node[card] (C4) [right=\thorpcarddistshort of C3] {3\ding{170}};
\node[card] (D1) [right=\thorpcarddist of C4] {4\ding{170}};
\node[card] (D2) [right=\thorpcarddistshort of D1] {4\ding{169}};
\node[card] (D3) [right=\thorpcarddistshort of D2] {4\ding{168}};
\node[card] (D4) [right=\thorpcarddistshort of D3] {4\ding{171}};

\foreach \z in {A, B, C} {
  \node (jm\z) [right=\thorpcarddisthalf of \z4] {};
  \node (jt\z) [above=\thorpcarddivbot of jm\z] {};
  \node (jb\z) [below=\thorpcarddivbot of jm\z] {};
}
\path[dashed,-]
  (jtA.center) edge[thick] node {} (jbA.center)
  (jtB.center) edge[thick] node {} (jbB.center)
  (jtC.center) edge[thick] node {} (jbC.center);

\node[draw,fill,circle,minimum width=0.1cm] (shuf1) [below=0.6cm of s4] {};

\path[-]
  (s1.south) edge node {} (shuf1.center)
  (h1.south) edge node {} (shuf1.center)
  (d1.south) edge node {} (shuf1.center)
  (c1.south) edge node {} (shuf1.center);
\path[->]
  (shuf1.center) edge node {} (A1.60)
  (shuf1.center) edge node {} (A2.60)
  (shuf1.center) edge node {} (A3.65)
  (shuf1.center) edge node {} (A4);
\end{tikzpicture}
\hspace{4em}
\begin{tikzpicture}
    \node[draw,thick,minimum width=2.0cm] (r0) at ($(0,0)$)  {$R_0$};
    \node[draw,thick,minimum width=2.0cm] (l0) [left of = r0, node distance=2.5cm] {$L_0$};
    \node (r0label) [above=0.1cm of r0] {$n/2$ bits};
    \node (l0label) [above=0.1cm of l0] {$n/2$ bits};
    \node[draw,thick,minimum width=0.75cm] (\prf) [below=0.75cm of r0.south east, anchor=north east]  {$\prf_{\prfkey}$};
    \node (xor) [XOR, below of = \prf, node distance = 0.75cm, scale=1.0] {};
    \node[minimum width=0.75cm] (junction0) [below=0.25cm of r0.south east, anchor=north east] {};
    \node (junction1) [below left= 0.3cm and -0.5cm of l0] {};
    \node (junction2) [below left= 1.75cm and -0.5cm of l0] {};
    \node (junction3) [left of = xor, node distance = 0.5cm] {};
    \path[->]
      (r0.333) edge[thick] node {} (\prf)
      (\prf) edge[thick] node {} (xor)
      (junction3.center) edge[thick] node {} (xor);
    \path[-]
      (l0.205) edge[thick] node {} (junction1.center)
      (junction0.center) edge[thick] node {} (junction2.center)
      (junction1.center) edge[thick] node {} (junction3.center);
    \node[draw,thick,minimum width=2.0cm] (r1) [below=1.0cm of \prf.south east, anchor=north east] {$R_1$};
    \node[draw,thick,minimum width=2.0cm] (l1) [left of = r1, node distance=2.5cm] {$L_1$};
    \path[->]
      (xor) edge[thick] node {} (r1.27)
      (junction2.center) edge[thick] node {} (l1.155);
\end{tikzpicture}
}}
\caption{Balanced Thorp shuffle (left).
The deck is split into equal sized piles. In turn, one card from each pile is drawn and together are randomly placed.
The resulting permutation matches that of a balanced Feistel network (right), where the randomness of the card placements corresponds to a possible Feistel round function.
}
\label{fig:balanced-card-shuffle}
\end{figure}
