%%%%%%%%%%%%%%%%%%%%%%%%%%%%%%%%%%%%%%%%%%%%%%%%%%%%%%%%%%%%%%%%%%%%%%%%%%%%%%%%
\section{Randomized and Nonce-Respecting Symmetric Encryption}
\label{sec:freqanalysis}


\fpage{.25}{
\underline{$\INDCPA^\advA_{\SE}$}\\[1pt]
$K \getsr \kg$\\
$b \getsr \bits$\\
$b' \getsr \advA^{\EncOracle}$\\
Ret $b'$\medskip

\underline{$\EncOracle(M_0,M_1)$}\\
If $|M_0| \ne |M_1|$ then\\
\myInd Ret $\bot$\\
$C \getsr \enc_K(M_b)$\\
Ret $C$
}


\bnm
\AdvINDCPA{\SE}{\advA} = 2\cdotsm\Prob{\INDCPA_\SE^\advA\Rightarrow\true} - 1
\enm

\bnm
\AdvINDCPA{\SE}{\advA} = 
    \left|\Prob{\INDCPA1_{\SE}^\advA\Rightarrow 1} - \Prob{\INDCPA0_{\SE}^\advA\Rightarrow1} \right| 
\enm


\begin{theorem}
Let $\SE$ be a symmetric encryption scheme. Let $\advA$ be any
$\INDCPA_\SE$-adversary making at most $q$ queries. 
We give an $\ROR_\SE$-adversary $\advB$ such that
\bnm
  \AdvINDCPA{\SE}{\advA} \le 2\cdotsm\AdvROR{\SE}{\advB}
\enm
Adversary $\advB$ makes at most $q$ 
queries and runs in time that of $\advA$.
\end{theorem}

\tnote{We'll possibly introduce the following at some point, but didn't need it
here yet.}
We sometimes use $\bigO$ notation to hide small values that can be derived from
proofs, but don't matter to the interpretation of the theorem. If we were to do
an asymptotic treatment, this would correspond to hiding constants, hence the
abuse of notation.  Thus in above theorem we would replace $q+3$ with
$\bigO(q)$. 


\fpage{.20}{
\underline{$\advB^{\Enc}$}\\[1pt]
$b \getsr \bits$\\
$b' \getsr \advA^\EncSim$\\
If $(b = b')$ then Ret 1\\
Ret 0\medskip

\underline{$\EncSim(M_0,M_1)$}\\
Return $\Enc(M_b)$
}

\begin{align*}
\AdvROR{\SE}{\advB} 
    &= \left|\Prob{\ROR1_\SE^\advB\Rightarrow 1} -
                                \Prob{\ROR0_\SE^\advB\Rightarrow 1}\right|\\
    &= \left|\Prob{\INDCPA_\SE^\advA\Rightarrow\true} - \frac{1}{2}\right|\\
    &= \left|\frac{1}{2} +
    \frac{1}{2}\cdot\AdvINDCPA{\SE}{\advA} - \frac{1}{2}\right|\\
    &= \frac{1}{2}\cdot\AdvINDCPA{\SE}{\advA}
\end{align*}



\fpage{.25}{
\underline{$\INDSIM1^\advA_{\SE}$}\\[1pt]
$K \getsr \kg$\\
$b' \getsr \advA^{\EncOracle}$\\
Ret $b'$\medskip

\underline{$\EncOracle(M)$}\\
$C \getsr \enc_K(M)$\\
Ret $C$
}

\fpage{.25}{
\underline{$\INDSIM0^{\advA,\simu}_{\SE}$}\\[1pt]
$b' \getsr \advA^{\EncOracle}$\\
Ret $b'$\medskip

\underline{$\EncOracle(M)$}\\
$C \getsr \simu(|M|)$\\
Ret $C$
}
\bnm
\AdvINDSIM{\SE,\simu}{\advA} = 
    \left|\Prob{\INDSIM1_{\SE}^\advA\Rightarrow 1} - \Prob{\INDSIM0_{\SE,\simu}^\advA\Rightarrow1} \right| 
\enm

\begin{theorem*}
Let $\SE$ be a symmetric encryption scheme. We give a simulator $\simu$ such
that for any  an
$\INDSIM_\SE$-adversary $\advA$, we can give 
a $\ROR_\SE$-adversary $\advB$ such that
\bnm
  \AdvINDSIM{\SE,\simu}{\advA} \le \AdvROR{\SE}{\advB} \;.
\enm
Adversary $\advB$ runs in time that of $\advA$ and makes the same number of
queries. Simulator $\simu$ requires just two operations.
\end{theorem*}

\begin{align*}
  \Prob{\INDSIM1_{\SE}^\advA} &= \Prob{\ROR1_{\SE}^\advB}\\ 
  \Prob{\INDSIM0_{\SE}^\advA} &= \Prob{\ROR0_{\SE}^\advB} 
\end{align*}

\fpage{.15}{
\underline{$\simu(\ell)$}\\
$C \getsr \bits^{\ctxtlen(\ell)}$\\
Ret $C$
}

\fpage{.15}{
\underline{$\advB^\Enc$}\\
$b' \getsr \advA^{\EncSim}$\\
Ret $b'$\medskip

\underline{$\EncSim(M)$}\\
Return $\Enc(M)$
}

Notation $X_1,\ldots,X_m \getparse{n} X$ takes a string $X$ and partitions it
into as many full $n$-bit blocks as possible, and lets $X_m$ be remaining bits,
thus $|X_m| = |X| \bmod n$. Also recall that we assume that  $X \oplus Y$ for
$|X| > |Y|$ 
first truncates $X$ to $|Y|$ bits, and returns the exclusive or of that
truncated string with $Y$. We similarly define the operation when $|X| < |Y|$.

\fpage{.25}{
\underline{$\Enc(K,M)$}\\
$\IV \getsr \bits^n$\\
$M_1,\ldots,M_m \getparse{n} M$\\
For $i = 1$ to $m$ do\\
\myInd $C_i \gets E_K(\IV \oplus \langle i\rangle) \oplus M_i$\\
Ret $\IV \concat C_1 \concat\cdots\concat C_m$\medskip

\underline{$\Dec(K,C)$}\\
If $|C| \le n$ then Ret $\bot$\\
$\IV,C_1,\ldots,C_m \getparse{n} C$\\
For $i = 1$ to $m$ do\\
\myInd $C_i \gets E_K(\IV \oplus \langle i\rangle) \oplus C_i$\\
Ret $M_1 \concat\cdots\concat M_m$
}

\begin{theorem*}
Let $\CTR$ be the CTR-mode symmetric encryption scheme built using blockcipher
$\cipherE\Colon\bits^k\times\bits^n\rightarrow\bits^n$. Let $\advA$ be an
$\ROR_\CTR$-adversary making at most~$q$ queries each totaling at most~$\sigma$
blocks. Then we give an $\PRF_\cipher$-adversary $\advB$ such
that
\bnm
  \AdvROR{\CTR}{\advA} \le \AdvPRF{\cipher}{\advB} + \frac{2\sigma q^2}{2^n}
\enm
Adversary~$\advB$ runs in time that of $\advA$ plus $\bigO(q\sigma)$ and makes at most
$\sigma$ queries.
\end{theorem*}


\begin{align*}
  \AdvROR{\CTR}{\advA} 
    &= \left| \Prob{\REAL_\CTR^\advA} - \Prob{\RAND_\CTR^\advA}\right|\\
    &= \left| \Prob{\G0} - \Prob{\RAND_\CTR^\advA}\right|\\
    &\le \left| \Prob{\G1} + \AdvPRF{\cipher}{\advB} - \Prob{\RAND_\CTR^\advA}\right|\\
    &= \left| \Prob{\G2} + \AdvPRF{\cipher}{\advB} - \Prob{\RAND_\CTR^\advA}\right|\\
    &\le \left| \Prob{\G3} + \Prob{\bad_3} \AdvPRF{\cipher}{\advB} - \Prob{\RAND_\CTR^\advA}\right|\\
    &= \left| \Prob{\G4} + \Prob{\bad_4} + \AdvPRF{\cipher}{\advB} - \Prob{\RAND_\CTR^\advA}\right|\\
    &= \Prob{\bad_4} + \AdvPRF{\cipher}{\advB}\\
    &\le \AdvPRF{\cipher}{\advB} + \frac{2q\sigma^2}{2^n}
\end{align*}

To analyze the setting of $\bad_4$ we, first, observe that the choices of $\IV$
are independent of the adversary's queries. For query $i$, let $m_i$ be the
length in blocks of that query and $\IV_i$ be the value $\IV$ chosen.
Let $X_{i,j} = \IV_i \oplus \langle j\rangle$ for $1 \le j \le m_i$. Then we are
asking whether there are any collisions among 
\begin{align*}
    &\IV_1 , \IV_1 \oplus \langle 1\rangle, \ldots , \IV_1 \oplus \langle m_1\rangle \\
    &\IV_2 , \IV_2 \oplus \langle 1\rangle, \ldots , \IV_2 \oplus \langle m_2\rangle \\
    &\phantom{\IV_2 , \IV_2 \oplus \langle 1\rangle, } \vdots\\
    &\IV_q , \IV_q \oplus \langle 1\rangle, \ldots , \IV_q \oplus \langle m_q\rangle \\
\end{align*}
If $m_i < 2^n-1$ for all queries then we can ignore wraparound effects, and so
no collisions on each row. Now consider any two pairs of rows $1 < i < j$. We
can consider $\IV_i$ to be fixed, and so a collision across rows occurs if
$\IV_j \in \{\IV_i + \alpha \;:\; -\sigma +1 \le \alpha \le \sigma-1\}$. This
occurs with probability $(2\sigma-1) / 2^n$. Since there are at most $q^2$ such
pairs, we have that the total probability of a collision is at most 
$2q^2\sigma / 2^n$.

%\begin{align*}
%  \Prob{\bad_4} = \Prob{\wedge_i X_i \in T_i  \lor X_2 \in T_i \cdots \lor X_{
%\end{align*}

\hfpagessss{.23}{.23}{.23}{.23}{
\underline{$\G0$}\\
$K \getsr \bits^k$\\
$b' \getsr \advA^\EncOracle$\\
Ret $b'$\medskip

\underline{$\EncOracle(M)$}\\
$\IV \getsr \bits^n$\\
$M_1,\ldots,M_m \getparse{n} M$\\
For $i = 1$ to $m$ do\\
\myInd $C_i \gets E_K(\IV \oplus \langle i\rangle) \oplus M_i$\\
Ret $\IV \concat C_1 \concat\cdots\concat C_m$
}{
\underline{$\G1$}\\
$\rho \getsr \Func(n,n)$\\
$b' \getsr \advA^\EncOracle$\\
Ret $b'$\medskip

\underline{$\EncOracle(M)$}\\
$\IV \getsr \bits^n$\\
$M_1,\ldots,M_m \getparse{n} M$\\
For $i = 1$ to $m$ do\\
\myInd $C_i \gets \rho(\IV \oplus \langle i\rangle) \oplus M_i$\\
Ret $\IV \concat C_1 \concat\cdots\concat C_m$
}{
\underline{\fbox{$\G2$} \;\;\; $\G3$}\\
$b' \getsr \advA^\EncOracle$\\
Ret $b'$\medskip

\underline{$\EncOracle(M)$}\\
$\IV \getsr \bits^n$\\
$M_1,\ldots,M_m \getparse{n} M$\\
For $i = 1$ to $m$ do\\
\myInd $P_i \getsr \bits^n$\\
\myInd If $\TabT[\IV\oplus \langle i \rangle] \ne \bot$ then\\
\myInd\myInd $\bad\gets\true$\\
\myInd\myInd \fbox{$C_i \gets \TabT[\IV \oplus \langle i\rangle]$}\\
\myInd $\TabT[\IV \oplus \langle i\rangle] \gets P_i $\\
\myInd $C_i \gets P_i \oplus M_i$\\
Ret $\IV \concat C_1 \concat\cdots\concat C_m$
}{
\underline{$\G4$}\\
$b' \getsr \advA^\EncOracle$\\
Ret $b'$\medskip

\underline{$\EncOracle(M)$}\\
$\IV \getsr \bits^n$\\
$M_1,\ldots,M_m \getparse{n} M$\\
For $i = 1$ to $m$ do\\
\myInd If $\TabT[\IV\oplus \langle i \rangle] \ne \bot$ then\\
\myInd\myInd $\bad\gets\true$\\
\myInd $\TabT[\IV \oplus \langle i\rangle] \gets 1 $\\
\myInd $C_i \getsr \bits^n$\\
Ret $\IV \concat C_1 \concat\cdots\concat C_m$
}
