%%%%%%%%%%%%%%%%%%%%%%%%%%%%%%%%%%%%%%%%%%%%%%%%%%%%%%%%%%%%%%%%%%%%%%%%%%%%%%%%
\section{Randomized and Nonce-Respecting Symmetric Encryption}
\label{sec:symenc}

Given some symmetric encryption scheme, in what sense is it secure?  To answer
this question in a formal manner, we give several security conditions below.

\subsection{Security Conditions}

\paragraph{Real-or-Random Indistinguishability (\INDRAND or \ROR)}

\fpage{.25}{
  \underline{$\RORreal^{\advA}_{\SEscheme}$}\\[1pt]
  $K \getsr \kg$ \\
  $b' \getsr \advA^{\CEnc}$ \\
  return $b'$ \\ \\
  \underline{$\SEenc(M)$} \\
  $C \getsr \enc_{K}(M)$ \\
  return $C$
}
\fpage{.25}{
  \underline{$\RORrand^{\advA}_{\SEscheme}$}\\[1pt]
  $b' \getsr \advA^{\SEenc}$ \\
  return $b'$ \\ \\ \\
  \underline{$\SEenc(M)$} \\
  $C \getsr \{0,1\}^{\ctxtlen(\absv{M})}$ \\
  return $C$
}

Intuitively, this condition captures the notion that ciphertexts ``look'' like
random bits under chosen-plaintext attacks. In the figure above, two
games are defined: in $\RORreal$, the oracle for the adversary actually
encrypts the adversary's chosen plaintexts, while in $\RORrand$ the oracle
just generates a random bitstring. The advantage for the adversary $\advA$ given
a symmetric encryption scheme $\SEscheme$ is

\bnm
\AdvROR{\SE}{\advA} =
\absv{\Prob{\RORreal_{\SE}^{\advA} \Rightarrow 1} - 
      \Prob{\RORrand_{\SE}^{\advA} \Rightarrow 1}}
\enm

Thus the adversary has high advantage when it can reliably distinguish the real
or random worlds.

\paragraph{Indistinguishability under Chosen-Plaintext Attack (\INDCPA)}


\fpage{.25}{
\underline{$\INDCPA^\advA_{\SE}$}\\[1pt]
$K \getsr \kg$\\
$b \getsr \bits$\\
$b' \getsr \advA^{\EncOracle}$\\
Ret $b = b'$\medskip

\underline{$\EncOracle(M_0,M_1)$}\\
If $|M_0| \ne |M_1|$ then\\
\myInd Ret $\bot$\\
$C \getsr \enc_K(M_b)$\\
Ret $C$
}

This condition captures the notion that, given a ciphertex, an adversary can't
infer which of two chosen generated the ciphertext under encryption with a
randomly sampled key. This means that ciphertexts don't leak information about
their messages. In the figure above, a secret challenge bit is drawn randomly,
and the adversary's oracle uses it to determine which one of two plaintexts
chosen by the adversary is encrypted and returned.  Because the adversary can
correctly guess the challenge bit with probability $1/2$ just by randomly
drawing from $\{0,1\}$, the advantage for adversary $\advA$ given symmetric
encryption scheme $\SEscheme$  is scaled as follows:

\bnm
\AdvINDCPA{\SE}{\advA} = 2\cdotsm\Prob{\INDCPA_\SE^\advA\Rightarrow\true} - 1
\enm

Thus the adversary has high advantage if it can reliably guess the challenge
bit.

Alternatively, if we define two games $\INDCPA1$ and $\INDCPA0$ where the
challenge bit is set to 1 and 0 respectively, 

\bnm
\AdvINDCPA{\SE}{\advA} = 
    \left|\Prob{\INDCPA1_{\SE}^\advA\Rightarrow 1} - \Prob{\INDCPA0_{\SE}^\advA\Rightarrow1} \right| 
\enm

\paragraph{Simulation-based security (\INDSIM)}

\fpage{.25}{
\underline{$\INDSIM1^\advA_{\SE}$}\\[1pt]
$K \getsr \kg$\\
$b' \getsr \advA^{\EncOracle}$\\
Ret $b'$\medskip

\underline{$\EncOracle(M)$}\\
$C \getsr \enc_K(M)$\\
Ret $C$
}

\fpage{.25}{
\underline{$\INDSIM0^{\advA,\simu}_{\SE}$}\\[1pt]
$b' \getsr \advA^{\EncOracle}$\\
Ret $b'$\medskip

\underline{$\EncOracle(M)$}\\
$C \getsr \simu(|M|)$\\
Ret $C$
}

Finally, this condition captures the notion that ``having ciphertext is as good
as not having ciphertext'' --- more specifically, ciphertexts of chosen
plaintexts of random keys are indistinguishable from the output of a simulator
that only knows about the length of the plaintext. In the figure above, two
games are defined: in $\INDSIM1$, the oracle for the adversary actually
encrypts the adversary's chosen plaintexts, while in $\INDSIM0$ the oracle
returns the output of simulator $\simu$. The advantage for the adversary
$\advA$ given symmetric encryption scheme $\SEscheme$ is

\bnm
\AdvINDSIM{\SE,\simu}{\advA} = 
    \left|\Prob{\INDSIM1_{\SE}^\advA\Rightarrow 1} - \Prob{\INDSIM0_{\SE,\simu}^\advA\Rightarrow1} \right| 
\enm

Thus the adversary has high advantage if it can distinguish between
actual ciphertexts and simulator output.

\subsection{Reductions and Separations}
\label{sec:randomenc-reduct}

We have discussed three security conditions: \INDRAND, \INDCPA, and \INDSIM.
What is the relationship between these? It turns out that \INDRAND is
strictly stronger than \INDCPA and \INDSIM, and that \INDCPA and \INDSIM are
equivalent. We give a series of reductions and counterexamples to prove this.

\subsubsection*{$\INDRAND \Rightarrow \INDCPA$}

\begin{theorem}
Let $\SE$ be a symmetric encryption scheme. Let $\advA$ be any
$\INDCPA_\SE$-adversary making at most $q$ queries. 
We give an $\ROR_\SE$-adversary $\advB$ such that
\bnm
  \AdvINDCPA{\SE}{\advA} \le 2\cdotsm\AdvROR{\SE}{\advB}
\enm
Adversary $\advB$ makes at most $q$ 
queries and runs in time that of $\advA$.
\end{theorem}

% commenting this out in case tom wants to put it back in
%
% \tnote{We'll possibly introduce the following at some point, but didn't need it
% here yet.}
% We sometimes use $\bigO$ notation to hide small values that can be derived from
% proofs, but don't matter to the interpretation of the theorem. If we were to do
% an asymptotic treatment, this would correspond to hiding constants, hence the
% abuse of notation.  Thus in above theorem we would replace $q+3$ with
% $\bigO(q)$. 

\paragraph{Proof.}

Construct $\advB$ as follows: take $\advA$ and provide it an $\SEencsim$
oracle with the same signature as $\advA$'s $\SEenc$ oracle (i.e., it takes two
messages $M_0, M_1$ as input and returns some ciphertext) that sends to
$\advB$'s oracle one of two messages according to some challenge bit $b$.

\fpage{.20}{
\underline{$\advB^{\Enc}$}\\[1pt]
$b \getsr \bits$\\
$b' \getsr \advA^\EncSim$\\
If $(b = b')$ then Ret 1\\
Ret 0\medskip

\underline{$\EncSim(M_0,M_1)$}\\
Return $\Enc(M_b)$
}

Notice that that $\advB$ playing the $\RORreal$ game is equivalent to 
$\advA$ playing the $\INDCPA$ game, such that

\bnm
\Prob{\RORreal_{\SEscheme}^{\advB} \Rightarrow 1} =
\Prob{\INDCPA_{\SEscheme}^{\advA} \Rightarrow \true}
\enm

Also notice that for $\advB$ playing the $\RORrand$ game, its oracle returns
random bitstrings instead of actual ciphertext such that $\advA$, used by
$\advB$, gets no information about the scheme whatsoever --- the bitstrings
are generated independently of the value of $b$. This implies that $\advA$
is making a random guess (``flipping a coin'') regarding the value of $b$,
thus $\Prob{\RORrand_{\SEscheme}^{\advB} \Rightarrow 1} = 1/2$.

With these two facts and some elementary algebra, we get

\begin{align*}
\AdvROR{\SE}{\advB} 
    &= \left|\Prob{\ROR1_\SE^\advB\Rightarrow 1} -
                                \Prob{\ROR0_\SE^\advB\Rightarrow 1}\right|\\
    &= \left|\Prob{\INDCPA_\SE^\advA\Rightarrow\true} - \frac{1}{2}\right|\\
    &= \left|\frac{1}{2} +
    \frac{1}{2}\cdot\AdvINDCPA{\SE}{\advA} - \frac{1}{2}\right|\\
    &= \frac{1}{2}\cdot\AdvINDCPA{\SE}{\advA}
\end{align*}

as needed. $\blacksquare$

\subsubsection*{$\INDRAND \Rightarrow \INDSIM$}

\begin{theorem*}
Let $\SE$ be a symmetric encryption scheme. We give a simulator $\simu$ such
that for any  an
$\INDSIM_\SE$-adversary $\advA$, we can give 
a $\ROR_\SE$-adversary $\advB$ such that
\bnm
  \AdvINDSIM{\SE,\simu}{\advA} \le \AdvROR{\SE}{\advB} \;.
\enm
Adversary $\advB$ runs in time that of $\advA$ and makes the same number of
queries. Simulator $\simu$ requires just two operations.
\end{theorem*}

\paragraph{Proof.}

Construct $\simu$ and $\advB$ as follows:

\fpage{.15}{
\underline{$\simu(\ell)$}\\
$C \getsr \bits^{\ctxtlen(\ell)}$\\
Ret $C$
}

\fpage{.15}{
\underline{$\advB^\Enc$}\\
$b' \getsr \advA^{\EncSim}$\\
Ret $b'$\medskip

\underline{$\EncSim(M)$}\\
Return $\Enc(M)$
}

The intuition is that the simulator draws random bits much like the
$\RORrand$ game and thus $\advA$  essentially is playing the
$\RORreal$ and $\RORrand$ games respectively for
$\INDSIM1$ and $\INDSIM0$, such that, given $\advB$ is just an elementary
wrapper over $\advA$,
$\Prob{\RORreal_{\SEscheme}^{\advB}} = \Prob{\INDSIM1_{\SEscheme}^{\advA}}$
and
$\Prob{\RORrand_{\SEscheme}^{\advB}} = \Prob{\INDSIM0_{\SEscheme}^{\advA}}$.
Thus

\begin{align*}
  \AdvROR{\SEscheme}{\advB} &=
  \absv{\Prob{\RORreal^{\advB}_{\SEscheme}} - \Prob{\RORrand^{\advB}_{\SEscheme}}} \\
  &= \absv{\Prob{\INDSIM1^{\advA}_{\SEscheme}} - \Prob{\INDSIM0^{\advA}_{\SEscheme}}} \\
  &= \AdvINDSIM{\SEscheme}{\advA} \\
  &\geq \AdvINDSIM{\SEscheme}{\advA}
\end{align*}

as needed. $\blacksquare$

\subsubsection*{$\INDCPA \not\Rightarrow \INDRAND$}

Intuitively, a scheme with ciphertexts indistinguishable from each other
doesn't necessarily have ciphertexts indistinguishable from random bits.
We show a separation by constructing a scheme $\overline{\SEscheme}$ and
$\INDRAND$-adversary $\advA$ such that for any $\INDCPA$-adversary $\advB$,

\bnm
\AdvROR{\overline{\SEscheme}}{\advA} \geq
\AdvINDCPA{\overline{\SEscheme}}{\advB}
\enm

\paragraph{Proof.}

Pick a scheme $\SEscheme = (\textbf{kg},\textbf{enc},\textbf{dec})$
such that for some $0 < p < 1$, $\adv_{\SEscheme}^{\INDCPA}(\advB) \leq p$
for any $\INDCPA$-adversary $\advB$.
Construct scheme
$\overline{\SEscheme} = (\textbf{kg},\overline{\textbf{enc}},\overline{\textbf{dec}})$
by padding ciphertexts generated by $\SEscheme$ with 0s. The width of the
padding, given by $n$, is calculated from $p$: specifically, $n$ must be picked
such that $n \geq f(p)$ for function $f$ defined below.
$\overline{\textbf{enc}}$ is defined below as well.
$\overline{\textbf{dec}}$ strips the padding from the ciphertext before
decrypting with $\textbf{dec}$.

\fpage{.25}{
  \underline{$\overline{\enc}_K(M)$}\\[1pt]
  $C \getsr \enc_K(M)$ \\
  return $0^n \parallel C$
}

Notice that the padding is the same for all ciphertexts, so
$\INDCPA$-adversaries does not learn any additional information from the
padding. This implies that for all $\INDCPA$-adversaries $\advB$,

\bnm
\AdvINDCPA{\overline{\SEscheme}}{\advB} = \AdvINDCPA{\SEscheme}{\advB}
\enm

Next we construct a $\INDRAND$-adversary $\advA$ that uses its $\Enc$ oracle to
encrypt a random bitstring and then checks whether the first $n$ bits of the
ciphertext is 0s; if it does, $\advA$ guesses it is in the real world.

\fpage{.25}{
  \underline{$\advA^{\SEenc}$}\\[1pt]
  $M \getsr \{0,1\}$ \\
  $C \gets \SEenc(M)$ \\
  if $C[1..n] = 0^n$ then \\
  \ind return $1$ \\
  else \\
  \ind return $0$
}

By construction of $\overline{\SEscheme}$, if $\advA$ is playing the
$\RORreal$-game then it always guesses it is in the real world.
If $\advA$ is playing the $\RORrand$-game, with probability $1 / 2^n$
(the probability of drawing a random bitstring whose first $n$ bits is all 0s)
it guesses that it is in the real world. Thus 

\bnm
\adv_{\overline{\SEscheme}}^{\INDRAND}(\advA) = 1 - \frac{1}{2^n}
\enm

Recall that $n$ is picked such that $n \geq f(n)$, where intuitively $f$ is a
lower bound on the width of padding required so that $\advA$ obtains high enough
advantage over any $\advB$. Define $f$ as

\bnm
f(p) \geq \log_2(1 - p)
\enm

We arrive at the definition of $f$ by solving for the following inequality:

\begin{align*}
  p &\leq 1 - 1 / 2^{f(p)} \\
  1 / 2^{f(p)} &\leq 1 - p \\
  \log_2(1 / 2^{f(p)}) &\leq \log_2(1 - p) \\
  - f(p) &\leq \log_2(1 - p) \\
  f(p) &\geq \log_2(1 - p)
\end{align*}

Putting it all together, for any $\INDCPA$-adversary $\advB$ we have

\bnm
  \AdvINDCPA{\overline{\SEscheme}}{\INDCPA}{\advB}
  = \AdvINDCPA{\SEscheme}{\advB} \\
  \leq p \\
  \leq 1 - 1 / 2^{f(p)} \\
  \leq 1 - 1 / 2^n \\
  = \AdvROR{\overline{\SEscheme}}{\advA}
\enm

as needed. $\blacksquare$

\subsubsection*{$\INDSIM \not\Rightarrow \INDRAND$}

This separation immediately follows from composing the reduction
$\INDCPA \Rightarrow \INDSIM$ (below) and separation
$\INDCPA \not\Rightarrow \INDRAND$.

\subsubsection*{$\INDCPA \Rightarrow \INDSIM$}

Let $\SEscheme$ be a symmetric encryption scheme. There is
a simulator $\simu$ such that for all $\INDSIM$-adversaries $\advA$ there is
a $\INDCPA$-adversary $\advB$ such that 

\bnm
\AdvINDSIM{\SEscheme}{\advA} \leq \AdvINDCPA{\SEscheme}{\advB}
\enm

\paragraph{Proof.}

We construct $\advB$ and $\simu$ as shown below.

\fpage{.25}{
  \underline{$\advB^{\Enc}$}\\[1pt]
  $a \gets \advA^{\EncSim}$\\
  return $a$\\
  \\
  \underline{$\EncSim(M)$}\\[1pt]
  $r \getsr \{0,1\}^{\absv{M}}$\\
  return $\Enc(r, M)$
}
\fpage{.25}{
  \underline{$\simu(\ell)$}\\[1pt]
  $r \getsr \{0,1\}^{\absv{M}}$\\
  return $r$
}


The intuition is that $\Enc$ chooses to encrypt either the plaintext chosen
by $\advA$ or a random bitstring. The former corresponds to the $\INDSIM1$
game while the latter corresponds to the $\INDSIM0$ game. The correspondence
for the former follows immediately because it is just encrypting the chosen
plaintext. The correspondence for the latter follows because the ciphertext
for a random bitstring can be treated as a random bitstring itself;
it has no information about the plaintext chosen by $\advA$ since it is
randomly and independently generated from the chosen plaintext.

To arrive at the advantage of $\advB$, we first calculate the probability
that it will win the $\INDCPA$ game, where $b$ is the secret challenge bit:

\begin{align*}
  \Prob{\INDCPA^{\advB}_{\SEscheme} \Rightarrow \ctrue}
  &= \Prob{b = 0 \wedge a = 0} + \Prob{b = 1 \wedge a = 1} \\
  &= \Prob{b = 0}\Prob{a = 0 \mid b = 0} + \Prob{b = 1}\Prob{a = 1 \mid b = 1} \\
  &= 1/2(1 - \Prob{\INDSIM0^{\advA}_{\SEscheme} \Rightarrow 1})
     + 1/2(\Prob{\INDSIM1^{\advA}_{\SEscheme} \Rightarrow 1}) \\
  &= 1/2(1 - \Prob{\INDSIM0^{\advA}_{\SEscheme} \Rightarrow 1}
          + \Prob{\INDSIM1^{\advA}_{\SEscheme} \Rightarrow 1})
\end{align*}

Thus

\begin{align*}
  \AdvINDCPA{\SEscheme}{\advB} &=
    2 \Prob{\INDCPA^{\advB}_{\SEscheme} \Rightarrow \ctrue} - 1 \\
  &= 2(1/2(1 - \Prob{\INDSIM0^{\advA}_{\SEscheme} \Rightarrow 1}
          + \Prob{\INDSIM1^{\advA}_{\SEscheme} \Rightarrow 1})) -1 \\
  &= \Prob{\INDSIM1^{\advA}_{\SEscheme} \Rightarrow 1})) -
     \Prob{\INDSIM0^{\advA}_{\SEscheme} \Rightarrow 1} \\
  &= \AdvINDSIM{\SEscheme}{\advA} \\
  &\geq \AdvINDSIM{\SEscheme}{\advA} \\
\end{align*}

as needed. $\blacksquare$


\subsubsection*{$\INDSIM \Rightarrow \INDCPA$}

Let $\SEscheme$ be a symmetric encryption scheme. For any $\INDCPA$-adversary
$\advA$, we construct a $\INDSIM$-adversary $\advB$ such that

\bnm
\AdvINDCPA{\SEscheme}{\advA} \leq \AdvINDSIM{\SEscheme}{\advB}
\enm

\paragraph{Proof.}

\fpage{.25}{
  \underline{$\simu(\ell)$}\\[1pt]
  $C \gets \{0,1\}^{\ctxtlen(\ell)}$ \\
  return $C$
}
\fpage{.25}{
  \underline{$\advB^{\SEenc}$}\\[1pt]
  $b \getsr \{0,1\}$ \\
  $b' \getsr \advA^{\SEencsim}$ \\
  return $b = b'$ \\ \\
  \underline{$\SEencsim(M_0, M_1)$}\\[1pt]
  return $\SEenc(M_b)$
}

This reduction is very similar to $\INDRAND \Rightarrow \INDCPA$.  The
constructed $\INDSIM$-adversary $\advB$ is essentially the same as the
constructed adversary for that reduction. Since $\advB$ playing the $\INDSIM1$
game is the same as $\advA$ playing the $\RORreal$ game,
$\Prob{\INDSIM1_{\SE}^{\advB} \Rightarrow 1} = \Prob{\RORreal_{\SE}^{\advA} \Rightarrow 1}$.
Notice that when $\advB$ plays the $\RORrand$ game its oracle returns random
bitstrings instead of actual ciphertext such that $\advA$, used by $\advB$,
gets no information about the scheme whatsoever.  Thus
$\Prob{\INDSIM0_{\SEscheme}^{\advB} \Rightarrow 1} = 1/2$.

Then

\begin{align*}
  \AdvINDSIM{\SEscheme}{\advB} &=
    \absv{\Prob{\INDSIM1^{\advB}_{\SEscheme} \Rightarrow 1} -
          \Prob{\INDSIM0^{\advB}_{\SEscheme} \Rightarrow 1}} \\
  &= \absv{\Prob{\INDCPA^{\advA}_{\SEscheme} \Rightarrow \true} - 1 / 2} \\
  &= \absv{1 / 2 + 1 / 2 \cdot \AdvINDCPA{\SEscheme}{\advA} - 1 / 2} \\
  &= 1 / 2 \cdot \AdvINDCPA{\SEscheme}{\advA} \\
  &\geq 1 / 2 \cdot \AdvINDCPA{\SEscheme}{\advA}
\end{align*}

as needed. $\blacksquare$

\subsection{Example: IND\$ for CTR mode}

\fpage{.25}{
\underline{$\Enc(K,M)$}\\
$\IV \getsr \bits^n$\\
$M_1,\ldots,M_m \getparse{n} M$\\
For $i = 1$ to $m$ do\\
\myInd $C_i \gets E_K(\IV \oplus \langle i\rangle) \oplus M_i$\\
Ret $\IV \concat C_1 \concat\cdots\concat C_m$\medskip

\underline{$\Dec(K,C)$}\\
If $|C| \le n$ then Ret $\bot$\\
$\IV,C_1,\ldots,C_m \getparse{n} C$\\
For $i = 1$ to $m$ do\\
\myInd $C_i \gets E_K(\IV \oplus \langle i\rangle) \oplus C_i$\\
Ret $M_1 \concat\cdots\concat M_m$
}

Notation $X_1,\ldots,X_m \getparse{n} X$ takes a string $X$ and partitions it
into as many full $n$-bit blocks as possible, and lets $X_m$ be remaining bits,
thus $|X_m| = |X| \bmod n$. Also recall that we assume that  $X \oplus Y$ for
$|X| > |Y|$ 
first truncates $X$ to $|Y|$ bits, and returns the exclusive or of that
truncated string with $Y$. We similarly define the operation when $|X| < |Y|$.

Given a block cipher $\varepsilon$, we can construct the $\CTR$-mode
symmetric encryption scheme as seen above. We then show that
this scheme is $\INDRAND$ secure.

\begin{theorem*}
Let $\CTR$ be the CTR-mode symmetric encryption scheme built using blockcipher
$\cipherE\Colon\bits^k\times\bits^n\rightarrow\bits^n$. Let $\advA$ be an
$\ROR_\CTR$-adversary making at most~$q$ queries each totaling at most~$\sigma$
blocks. Then we give an $\PRF_\cipher$-adversary $\advB$ such
that
\bnm
  \AdvROR{\CTR}{\advA} \le \AdvPRF{\cipher}{\advB} + \frac{2\sigma q^2}{2^n}
\enm
Adversary~$\advB$ runs in time that of $\advA$ plus $\bigO(q\sigma)$ and makes at most
$\sigma$ queries.
\end{theorem*}

\paragraph{Proof.}

The proof is a game-hopping argument, as seen below.

\hfpagessss{.23}{.23}{.23}{.23}{
\underline{$\G0$}\\
$K \getsr \bits^k$\\
$b' \getsr \advA^\EncOracle$\\
Ret $b'$\medskip

\underline{$\EncOracle(M)$}\\
$\IV \getsr \bits^n$\\
$M_1,\ldots,M_m \getparse{n} M$\\
For $i = 1$ to $m$ do\\
\myInd $C_i \gets E_K(\IV \oplus \langle i\rangle) \oplus M_i$\\
Ret $\IV \concat C_1 \concat\cdots\concat C_m$
}{
\underline{$\G1$}\\
$\rho \getsr \Func(n,n)$\\
$b' \getsr \advA^\EncOracle$\\
Ret $b'$\medskip

\underline{$\EncOracle(M)$}\\
$\IV \getsr \bits^n$\\
$M_1,\ldots,M_m \getparse{n} M$\\
For $i = 1$ to $m$ do\\
\myInd $C_i \gets \rho(\IV \oplus \langle i\rangle) \oplus M_i$\\
Ret $\IV \concat C_1 \concat\cdots\concat C_m$
}{
\underline{\fbox{$\G2$} \;\;\; $\G3$}\\
$b' \getsr \advA^\EncOracle$\\
Ret $b'$\medskip

\underline{$\EncOracle(M)$}\\
$\IV \getsr \bits^n$\\
$M_1,\ldots,M_m \getparse{n} M$\\
For $i = 1$ to $m$ do\\
\myInd $P_i \getsr \bits^n$\\
\myInd If $\TabT[\IV\oplus \langle i \rangle] \ne \bot$ then\\
\myInd\myInd $\bad\gets\true$\\
\myInd\myInd \fbox{$C_i \gets \TabT[\IV \oplus \langle i\rangle]$}\\
\myInd $\TabT[\IV \oplus \langle i\rangle] \gets P_i $\\
\myInd $C_i \gets P_i \oplus M_i$\\
Ret $\IV \concat C_1 \concat\cdots\concat C_m$
}{
\underline{$\G4$}\\
$b' \getsr \advA^\EncOracle$\\
Ret $b'$\medskip

\underline{$\EncOracle(M)$}\\
$\IV \getsr \bits^n$\\
$M_1,\ldots,M_m \getparse{n} M$\\
For $i = 1$ to $m$ do\\
\myInd If $\TabT[\IV\oplus \langle i \rangle] \ne \bot$ then\\
\myInd\myInd $\bad\gets\true$\\
\myInd $\TabT[\IV \oplus \langle i\rangle] \gets 1 $\\
\myInd $C_i \getsr \bits^n$\\
Ret $\IV \concat C_1 \concat\cdots\concat C_m$
}

First, notice that $G0$ game is the same as the $\RORreal$ game for $\INDRAND$
security, and thus

\bnm
\Prob{\RORreal^{\advA}_{\CTR} \Rightarrow 1} = \Prob{G0 \Rightarrow 1}
\enm

Next, notice that $G0$ corresponds to $\textup{PRF1}$ and $G1$ corresponds to
$\textup{PRF0}$, in the sense that in $G0$ the message block is XORed with the
encryption of the IV and the block index, while in $G1$ the message block is
XORed with the output of a random function with the IV and block index as input.
It is then trivial to construct a $\PRF$-adversary $\advB$ whose advantage 
is

\begin{align*}
  \absv{\Prob{G0 \Rightarrow 1} - \Prob{G1 \Rightarrow 1}} = \AdvPRF{\varepsilon}{\advB} \\
  \Prob{G0 \Rightarrow 1} \leq \Prob{G1 \Rightarrow 1} + \AdvPRF{\varepsilon}{\advB}
\end{align*}

Next, since in $G2$ the value in $\TabT$ is re-used in the event of a collision
and is randomly selected at each sampled point, it essentially acts a random
function, which implies
$\Prob{G1 \Rightarrow 1} = \Prob{G2 \Rightarrow 1}$.

Since $G2$ and $G3$ are identical-until-bad, by the fundamental lemma of
game playing we have

\bnm
\Prob{G2 \Rightarrow 1} \leq \Prob{G3 \Rightarrow 1} + \Prob{\bad_3}
\enm

The only difference between $G3$ and $G4$ is that $G3$ derives ciphertext
using one-time pads and $G4$ derives ciphertext from random strings.
The distribution of ciphertexts for one-time pads is uniform, thus making
it equal to the distribution of random strings. This implies that the adversary
should not be able to distinguish between the two, so
$\Prob{G3 \Rightarrow 1} = \Prob{G4 \Rightarrow 1}$.
Furthermore, notice that $\bad$ is set in the same conditions as $G3$ and $G4$,
so $\Prob{\bad_3} = \Prob{\bad_4}$. Then

\bnm
\Prob{G3 \Rightarrow 1} + \Prob{\bad_3} = \Prob{G4 \Rightarrow 1} + \Prob{\bad_4}
\enm

Next, since $G4$ assigns the ciphertext to random bitstrings,

\bnm
\Prob{G4 \Rightarrow 1} = \Prob{\RORrand^{\advA}_{\CTR} \Rightarrow 1}
\enm

To analyze the setting of $\bad_4$ we, first, observe that the choices of $\IV$
are independent of the adversary's queries. For query $i$, let $m_i$ be the
length in blocks of that query and $\IV_i$ be the value $\IV$ chosen.
Let $X_{i,j} = \IV_i \oplus \langle j\rangle$ for $1 \le j \le m_i$. Then we are
asking whether there are any collisions among 
\begin{align*}
    &\IV_1 , \IV_1 \oplus \langle 1\rangle, \ldots , \IV_1 \oplus \langle m_1\rangle \\
    &\IV_2 , \IV_2 \oplus \langle 1\rangle, \ldots , \IV_2 \oplus \langle m_2\rangle \\
    &\phantom{\IV_2 , \IV_2 \oplus \langle 1\rangle, } \vdots\\
    &\IV_q , \IV_q \oplus \langle 1\rangle, \ldots , \IV_q \oplus \langle m_q\rangle \\
\end{align*}
If $m_i < 2^n-1$ for all queries then we can ignore wraparound effects, and so
no collisions can occur on each row. Now consider any two pairs of rows $1 < i
< j$. We can consider $\IV_i$ to be fixed, and so a collision across rows
occurs if $\IV_j \in \{\IV_i + \alpha \;:\; -\sigma +1 \le \alpha \le
\sigma-1\}$. This occurs with probability $(2\sigma-1) / 2^n$. Since there are
at most $q^2$ such pairs, we have that the total probability of a collision is
at most $2q^2\sigma / 2^n$. See Lemma 5.18 of Bellare-Rogaway for more details.

Putting all of these together, we arrive at

\begin{align*}
  \AdvROR{\CTR}{\advA} 
    &= \left| \Prob{\REAL_\CTR^\advA} - \Prob{\RAND_\CTR^\advA}\right|\\
    &= \left| \Prob{\G0} - \Prob{\RAND_\CTR^\advA}\right|\\
    &\le \left| \Prob{\G1} + \AdvPRF{\cipher}{\advB} - \Prob{\RAND_\CTR^\advA}\right|\\
    &= \left| \Prob{\G2} + \AdvPRF{\cipher}{\advB} - \Prob{\RAND_\CTR^\advA}\right|\\
    &\le \left| \Prob{\G3} + \Prob{\bad_3} \AdvPRF{\cipher}{\advB} - \Prob{\RAND_\CTR^\advA}\right|\\
    &= \left| \Prob{\G4} + \Prob{\bad_4} + \AdvPRF{\cipher}{\advB} - \Prob{\RAND_\CTR^\advA}\right|\\
    &= \Prob{\bad_4} + \AdvPRF{\cipher}{\advB}\\
    &\le \AdvPRF{\cipher}{\advB} + \frac{2q\sigma^2}{2^n}
\end{align*}

as needed. $\blacksquare$


%\begin{align*}
%  \Prob{\bad_4} = \Prob{\wedge_i X_i \in T_i  \lor X_2 \in T_i \cdots \lor X_{
%\end{align*}

