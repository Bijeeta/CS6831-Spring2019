\section{Randomized Encryption}
\label{sect:randomenc}

\subsection{Security Conditions}

\paragraph{Real-or-Random Indistinguishability (\INDRAND)}

Intuitively, this condition captures the notion that ciphertext ``look'' like
random bits under chosen-plaintext attacks. In \cref{fig:ind-rand}, two
games are defined: in $\RORreal$, the oracle for the adversary actually
encrypts the adversary's chosen plaintexts, while in $\RORrand$ the oracle
just generates a random bitstring. The advantage for the adversary $\advA$ given
symmetric encryption scheme $\SEscheme$ is

$$
\adv_{\SE}^{\INDRAND}(\advA) =
\absv{\Prob{\RORreal_{\SE}^{\advA} \Rightarrow 1} - 
      \Prob{\RORrand_{\SE}^{\advA} \Rightarrow 1}}
$$

Thus the adversary has high advantage when it can guess correctly whether it is
in the ``real world'' or the ``random world''.


\begin{figure}[p]
	\centering
	\fpage{.25}{
    \underline{$\RORreal^{\advA}_{\SEscheme}$}\\[1pt]
		$K \getsr \kg$ \\
    $b' \getsr \advA^{\CEnc}$ \\
    return $b'$ \\ \\
    \underline{$\SEenc(M)$} \\
    $C \getsr \enc_{K}(M)$ \\
    return $C$
	}
	\fpage{.25}{
    \underline{$\RORrand^{\advA}_{\SEscheme}$}\\[1pt]
    $b' \getsr \advA^{\SEenc}$ \\
    return $b'$ \\ \\ \\
    \underline{$\SEenc(M)$} \\
    $C \getsr \{0,1\}^{\ctxtlen(\absv{M})}$ \\
    return $C$
	}
  \caption{Random-or-real indistinguishability (\INDRAND)}
	\label{fig:ind-rand}
\end{figure} 


\paragraph{Indistinguishability under Chosen-Plaintext Attack (\INDCPA)}

This condition captures the notion that, given two plaintexts, we can't infer
which of two plaintexts were encrypted. This means that ciphertexts don't leak
information about their messages. In \cref{fig:ind-cpa}, a challenge bit is
drawn randomly, and the adversary's oracle uses it to determine which one
of two plaintexts chosen by the adversary is encrypted and returned.
Because the adversary can correctly guess the challenge bit with probability
$1/2$ just by randomly drawing from $\{0,1\}$, 
the advantage for adversary $\advA$ given symmetric encryption scheme
$\SEscheme$  is scaled as follows:

$$
\adv_{\SE}^{\INDCPA}(\advA) = \Prob{\INDCPA}(\advA) - \frac{1}{2}
$$

Thus the adversary has high advantage if it can reliably guess the challenge
bit.

\begin{figure}[p]
	\centering
	\fpage{.25}{
    \underline{$\INDCPA^{\advA}_{\SEscheme}$}\\[1pt]
		$K  \getsr \kg$ \\
    $b  \getsr \{0,1\}$ \\
    $b' \getsr \advA^{\SEenc}$ \\ \\
    \underline{$\SEenc(M_0,M_1)$} \\
    if $\absv{M_0} \neq \absv{M_1}$ then \\
    \ind return $\bot$ \\
    $C \getsr \enc_K(M_b)$ \\
    return $C$
	}
  \caption{Indistinguishability under Chosen-Plaintext Attack (\INDCPA)}
	\label{fig:ind-cpa}
\end{figure} 

\paragraph{Simulation-based security (\INDSIM)}

Finally, this condition captures the notion that
''having ciphertext is as good as not having ciphertext'' --- more specifically,
ciphertexts are indistinguishable from the output of a simulator that only
knows about the length of the plaintext.
In \cref{fig:ind-sim}, two games are defined: in $\INDSIM1$, the oracle for the
adversary actually encrypts the adversary's chosen plaintexts, while in
$\INDSIM0$ the oracle returns the output of simulator $\simu$.
The advantage for the adversary $\advA$ given symmetric
encryption scheme $\SEscheme$ is

$$
\adv_{\SE}^{\INDRAND}(\advA) =
\absv{\Prob{\INDSIM1{\SE}^{\advA} \Rightarrow 1} - 
      \Prob{\INDSIM0{\SE}^{\advA} \Rightarrow 1}}
$$

Thus the adversary has high advantage if it can distinguish between
actual ciphertexts and simulator output.

\begin{figure}[p]
	\centering
	\fpage{.25}{
    \underline{$\INDSIM1$}\\[1pt]
		$K  \getsr \kg$ \\
    $b'  \getsr \advA^{\SEenc}$ \\
    return $b'$ \\ \\
    \underline{$\SEenc(M)$} \\
    $C \getsr \enc_{K}(M)$ \\
    return $C$
  }
	\fpage{.25}{
    \underline{$\INDSIM0^{\advA,\simu}_{\SEscheme}$}\\[1pt]
    $b'  \getsr \advA^{\SEenc}$ \\
    return $b'$ \\ \\
    \underline{$\SEenc(M)$} \\
    $C \getsr \simu(\absv{M})$ \\
    return $C$
  }
  \caption{Simulation-based security (\INDSIM)}
	\label{fig:ind-sim}
\end{figure} 

While they are intuitive, in practice all of these security conditions are
inadequate because they don't resist active attacks that change ciphertext.

\subsection{Reductions}
\label{sec:randomenc-reduct}

We have discussed three security conditions: \INDRAND, \INDCPA, and \INDSIM.
What is the relationship between these? It turns out that \INDRAND is
strictly stronger than \INDCPA and \INDSIM, and that \INDCPA and \INDSIM are
equivalent. We give a series of reductions and counterexamples to prove this.

\subsubsection*{$\INDRAND \Rightarrow \INDCPA$}

\begin{figure}[p]
	\centering
	\fpage{.25}{
    \underline{$\advB^{\SEenc}$}\\[1pt]
    $b \getsr \{0,1\}$ \\
    $b' \getsr \advA^{\SEencsim}$ \\
    return $b = b'$ \\ \\
    \underline{$\SEencsim(M_0, M_1)$}\\[1pt]
    return $\SEenc(M_b)$
	}
  \caption{$\advB^{\SEenc}$ in $\INDRAND \Rightarrow \INDCPA$}
  \label{fig:indcpa-indrand-adv}
\end{figure}

\begin{theorem}
\label{thm:reduct-indcpa-indrand}

Let $\SEscheme$ be a symmetric encryption scheme. For any adversary $\advA$ for
  $\INDCPA_\SEscheme$, we can construct adversary $\advB$ for
  $\INDRAND_\SEscheme$ such that

$$
\adv^{\INDCPA}_{\SEscheme}(\advA) \leq 2 \cdot \adv^{\INDRAND}_{\SEscheme}(\advB)
$$
\end{theorem}

\paragraph{Proof.}
Construct $\advB$ as follows: take $\advA$ and provide it an $\SEencsim$
oracle with the same signature as $\advA$'s $\SEenc$ oracle (i.e., it takes two
messages $M_0, M_1$ as input and returns some ciphertext) that encrypts one
of two messages according to some challenge bit $b$. The code for adversary
$\advB$ is specified in~\cref{fig:indcpa-indrand-adv}.

Notice that that $\advB$ playing the $\RORreal$ game is equivalent to 
$\advA$ playing the $\INDCPA$ game, such that

$$
\Prob{\RORreal_{\SEscheme}^{\advB} \Rightarrow 1} =
\Prob{\INDCPA_{\SEscheme}^{\advA} \Rightarrow \true}
$$

Also notice that for $\advB$ playing the $\RORrand$ game, its oracle returns
random bitstrings instead of actual ciphertext such that $\advA$, used by
$\advB$, gets no information about the scheme whatsoever.  Thus
$\Prob{\RORrand_{\SEscheme}^{\advB} \Rightarrow 1} = 1/2$.

With these two facts and some elementary algebra, we get

\begin{align*}
  \adv^{\INDRAND}_{\SEscheme}(\advB) &=
    \absv{\Prob{\RORreal}^{\advB}_{\SEscheme} \Rightarrow 1 - 
          \Prob{\RORrand^{\advB}_{\SEscheme} \Rightarrow 1}} \\
  &= \absv{\Prob{\INDCPA^{\advA}_{\SEscheme} \Rightarrow \true} - 1 / 2} \\
  &= \absv{1 / 2 + 1 / 2 \cdot \adv^{\INDCPA}_{\SEscheme}(\advA) - 1 / 2} \\
  &= 1 / 2 \cdot \adv^{\INDCPA}_{\SEscheme}(\advA) \\
  &\geq 1 / 2 \cdot \adv^{\INDCPA}_{\SEscheme}(\advA)
\end{align*}

Thus proving the inequality. $\blacksquare$

\subsubsection*{$\INDRAND \Rightarrow \INDSIM$}

\begin{figure}[p]
	\centering
	\fpage{.25}{
    \underline{$\simu(\ell)$}\\[1pt]
    $C \getsr \{0,1\}^{\ctxtlen(\ell)}$ \\
    return $C$
	}
	\fpage{.25}{
    \underline{$\advB^{\Enc}$}\\[1pt]
    $b \getsr \advA^{\Enc}$ \\
    return $b$
	}
  \caption{$\simu$ and $\advB$ for $\INDRAND \Rightarrow \INDSIM$}
  \label{fig:rand-sim-reduct}
\end{figure} 

\begin{theorem}
Let $\SEscheme$ be a symmetric encryption scheme. There is a simulator $\simu$
such that for any $\INDSIM_{\SEscheme}$-adversary $\advA$, there is
an $\INDRAND_{\SEscheme}$-adversary $\advB$ such that
$$
\adv^{\INDSIM}_{\SEscheme,\simu}(\advA) \leq \adv^{\INDRAND}_{\SEscheme}(\advB)
$$
\label{thm:indsim-indrand-reduct}
\end{theorem}

\paragraph{Proof.}
We define $\simu$ and $\advB$ in \cref{fig:rand-sim-reduct}.
The intuition is that the simulator draws random bits much like the
$\RORrand$ game and thus $\advA$  essentially is playing the
$\RORreal$ and $\RORrand$ games respectively for
$\INDSIM1$ and $\INDSIM0$, such that, given $\advB$ is just an elementary
wrapper over $\advA$,
$\Prob{\RORreal_{\SEscheme}^{\advB}} = \Prob{\INDSIM1_{\SEscheme}^{\advA}}$
and
$\Prob{\RORrand_{\SEscheme}^{\advB}} = \Prob{\INDSIM0_{\SEscheme}^{\advA}}$.
Thus

\begin{align*}
\adv^{\INDRAND}_{\SEscheme}(\advB) &=
  \absv{\Prob{\RORreal^{\advB}_{\SEscheme}} - \Prob{\RORrand^{\advB}_{\SEscheme}}} \\
  &= \absv{\Prob{\INDSIM1^{\advA}_{\SEscheme}} - \Prob{\INDSIM0^{\advA}_{\SEscheme}}} \\
  &= \adv^{\INDSIM}_{\SEscheme}(\advA) \\
  &\geq \adv^{\INDSIM}_{\SEscheme}(\advA)
\end{align*}

Thus the inequality holds. $\blacksquare$

\subsubsection*{$\INDCPA \not\Rightarrow \INDRAND$}

\begin{figure}[p]
	\centering
	\fpage{.25}{
    \underline{$\overline{\enc}_K(M)$}\\[1pt]
    $C \getsr \enc_K(M)$ \\
    return $0^n \parallel C$
	}
	\fpage{.25}{
    \underline{$\advA^{\SEenc}$}\\[1pt]
    $M \getsr \{0,1\}$ \\
    $C \gets \SEenc(M)$ \\
    if $C[1..n] = 0^n$ then \\
    \ind return $1$ \\
    else \\
    \ind return $0$
	}
  \caption{$\overline{\enc}$ and $\advA$ for $\INDCPA \not\Rightarrow \INDRAND$}
  \label{fig:cpa-rand-separation}
\end{figure}

Intuitively, a scheme with ciphertexts indistinguishable from each other
doesn't necessarily have ciphertexts indistinguishable from random bits.
We show a separation by constructing a scheme $\overline{\SEscheme}$ and
$\INDRAND$-adversary $\advA$ such that for any $\INDCPA$-adversary $\advB$,

$$
\adv_{\overline{\SEscheme}}^{\INDRAND}(\advA) \geq
\adv_{\overline{\SEscheme}}^{\INDCPA}(\advB)
$$

\paragraph{Proof.}

Pick a scheme $\SEscheme = (\textbf{kg},\textbf{enc},\textbf{dec})$
such that for some $0 < p < 1$, $\adv_{\SEscheme}^{\INDCPA}(\advB) \leq p$
for any $\INDCPA$-adversary $\advB$.
Construct scheme
$\overline{\SEscheme} = (\textbf{kg},\overline{\textbf{enc}},\overline{\textbf{dec}})$
by padding ciphertexts generated by $\SEscheme$ with 0s. The width of the
padding, given by $n$, is calculated from $p$: specifically, $n$ must be picked
such that $n \geq f(p)$ for function $f$ defined below.
$\overline{\textbf{enc}}$ is defined in \cref{fig:cpa-rand-separation};
$\overline{\textbf{dec}}$ strips the padding from the ciphertext before
decrypting with $\textbf{dec}$.

Notice that the padding is the same for all ciphertexts, so
$\INDCPA$-adversaries does not learn any additional information from the
padding. This implies that for all $\INDCPA$-adversaries $\advB$,

$$
\adv_{\overline{\SEscheme}}^{\INDCPA}(\advB) = \adv_{\SEscheme}^{\INDCPA}(\advB)
$$

Next we construct a $\INDRAND$-adversary $\advA$ (\cref{fig:cpa-rand-separation})
that uses its $\Enc$ oracle to encrypt a random bitstring and then checks
whether the first $n$ bits of the ciphertext is 0s;
if it does, $\advA$ guesses it is in the real world.
By construction of $\overline{\SEscheme}$, if $\advA$ is playing the
$\RORreal$-game then it always guesses it is in the real world.
If $\advA$ is playing the $\RORrand$-game, with probability $1 / 2^n$
(the probability of drawing a random bitstring whose first $n$ bits is all 0s)
it guesses that it is in the real world. Thus 

$$
\adv_{\overline{\SEscheme}}^{\INDRAND}(\advA) = 1 - \frac{1}{2^n}
$$

Recall that $n$ is picked such that $n \geq f(n)$, where intuitively $f$ is a
lower bound on the width of padding required so that $\advA$ obtains high enough
advantage over any $\advB$. Define $f$ as

$$
f(p) \geq \log_2(1 - p)
$$

We arrive at the definition of $f$ by solving for the following inequality:

\begin{align*}
  p &\leq 1 - 1 / 2^{f(p)} \\
  1 / 2^{f(p)} &\leq 1 - p \\
  \log_2(1 / 2^{f(p)}) &\leq \log_2(1 - p) \\
  - f(p) &\leq \log_2(1 - p) \\
  f(p) &\geq \log_2(1 - p)
\end{align*}

Putting it all together, for any $\INDCPA$-adversary $\advB$ we have

$$
  \adv_{\overline{\SEscheme}}^{\INDCPA}(\advB)
  = \adv_{\SEscheme}^{\INDCPA}(\advB) \\
  \leq p \\
  \leq 1 - 1 / 2^{f(p)} \\
  \leq 1 - 1 / 2^n \\
  = \adv_{\overline{\SEscheme}}^{\INDRAND}(\advA)
$$

as needed. $\blacksquare$

\subsubsection*{$\INDSIM \not\Rightarrow \INDRAND$}

This separation immediately follows from composing the reduction
$\INDCPA \Rightarrow \INDSIM$ (below) and separation
$\INDCPA \not\Rightarrow \INDRAND$.

\subsubsection*{$\INDCPA \Rightarrow \INDSIM$}

TODO: show for any symmetric encryption scheme $\SEscheme$, there is
a simulator $\simu$ such that for all $\INDSIM$-adversaries $\advA$ there is
a $\INDCPA$-adversary $\advB$ such that 

$$
\adv_{\SEscheme}^{\INDSIM}(\advA) \leq \adv_{\SEscheme}^{\INDCPA}(\advB)
$$

\subsubsection*{$\INDSIM \Rightarrow \INDCPA$}

\begin{figure}[p]
	\centering
	\fpage{.25}{
    \underline{$\simu(\ell)$}\\[1pt]
    $C \gets$
	}
	\fpage{.25}{
    \underline{$\advB^{\SEenc}$}\\[1pt]
    $b \getsr \{0,1\}$ \\
    $b' \getsr \advA^{\SEencsim}$ \\
    return $b = b'$ \\ \\
    \underline{$\SEencsim(M_0, M_1)$}\\[1pt]
    return $\SEenc(M_b)$
	}
  \caption{$\simu$ and $\advB$ in $\INDSIM \Rightarrow \INDCPA$}
  \label{fig:indsim-indrand-reduct}
\end{figure}

Let $\SEscheme$ be a symmetric encryption scheme. For any $\INDCPA$-adversary
$\advA$, we construct a $\INDSIM$-adversary $\advB$ such that

$$
\adv_{\SEscheme}^{\INDCPA}(\advA) \leq \adv_{\SEscheme}^{\INDSIM}(\advB)
$$

\paragraph{Proof.}

This reduction is very similar to $\INDRAND \Rightarrow \INDCPA$.
The constructed $\INDSIM$-adversary $\advB$ (\cref{fig:indsim-indrand-reduct})
is essentially the same as the constructed adversary for that reduction
(\cref{fig:indcpa-indrand-adv}). Since $\advB$ playing the $\INDSIM1$ game
is the same as $\advA$ playing the $\RORreal$ game,
$\Prob{\INDSIM1_{\SE}^{\advB} \Rightarrow 1} = \Prob{\RORreal_{\SE}^{\advA}}$.
Notice that when $\advB$ plays the $\RORrand$ game its oracle returns
random bitstrings instead of actual ciphertext such that $\advA$, used by
$\advB$, gets no information about the scheme whatsoever. Thus
$\Prob{\INDSIM0_{\SEscheme}^{\advB} \Rightarrow 1} = 1/2$.

Then

\begin{align*}
  \adv^{\INDSIM}_{\SEscheme}(\advB) &=
    \absv{\Prob{\INDSIM1^{\advB}_{\SEscheme} \Rightarrow 1} -
          \Prob{\INDSIM0^{\advB}_{\SEscheme} \Rightarrow 1}} \\
  &= \absv{\Prob{\INDCPA^{\advA}_{\SEscheme} \Rightarrow \true} - 1 / 2} \\
  &= \absv{1 / 2 + 1 / 2 \cdot \adv^{\INDCPA}_{\SEscheme}(\advA) - 1 / 2} \\
  &= 1 / 2 \cdot \adv^{\INDCPA}_{\SEscheme}(\advA) \\
  &\geq 1 / 2 \cdot \adv^{\INDCPA}_{\SEscheme}(\advA)
\end{align*}

as needed. $\blacksquare$


\subsection{Example: IND\$ for CTR mode}

Given a block cipher $\varepsilon$, we can construct the $\CTR$-mode
symmetric encryption scheme as seen in \cref{fig:ctr-mode}. We then show that
this scheme is $\INDRAND$ secure.

\begin{figure}[p]
	\centering
	\fpage{.40}{
    \underline{$\textbf{Enc}(K,M)$}\\[1pt]
    $IV \getsr \{0,1\}^n$ \\
    $M_1, \ldots, M_m \gets^{m} M$
    for $i = 1$ to $m$ do \\
    \ind $C_i \gets E_K(IV \bigoplus \langle i \rangle) \bigoplus M_i$ \\
    return $IV \parallel C_1 \parallel \cdots \parallel C_m$ \\
    \\
    \underline{$\textbf{Dec}(K,C)$}\\[1pt]
    if $\absv{C} \leq n$ then return $\bot$ \\
    $IV, C_1, \ldots, C_m \gets^m C$ \\
    for $i = 1$ to $m$ do \\
    \ind $M_i \gets^m D_K(IV \bigoplus \langle i \rangle) \bigoplus C_i$ \\
    return $M_1 \parallel \cdots \parallel M_m$
	}
  \caption{CTR-mode for a block cipher}
  \label{fig:ctr-mode}
\end{figure}

\begin{figure}[p]
	\centering
	\fpage{.25}{
    \underline{G0}\\[1pt]
    $K \getsr \{0,1\}^k$ \\
    $b' \gets \advA^{\SEenc}$ \\
    return $b'$ \\
    \\
    \underline{$\SEenc(M)$}\\[1pt]
    $IV \getsr \{0,1\}^n$ \\
    $M_1, \ldots, M_m \gets^{m} M$ \\
    for $i = 1$ to $m$ do \\
    \ind $C_i \gets E_K(IV \bigoplus \langle i \rangle) \bigoplus M_i$ \\
    return $IV \parallel C_1 \parallel \cdots \parallel C_m$
	}
	\fpage{.25}{
    \underline{G1}\\[1pt]
    $\rho \getsr \Func(n,n)$ \\
    $b' \gets \advA^{\SEenc}$ \\
    return $b'$ \\
    \\
    \underline{$\SEenc(M)$}\\[1pt]
    $IV \getsr \{0,1\}^n$ \\
    $M_1, \ldots, M_m \gets^{m} M$ \\
    for $i = 1$ to $m$ do \\
    \ind $C_i \gets \rho(IV \bigoplus \langle i \rangle) \bigoplus M_i$ \\
    return $IV \parallel C_1 \parallel \cdots \parallel C_m$
	}
	\fpage{.25}{
    \underline{{\color{blue}G2} \; G3}\\[1pt]
    $b' \gets \advA^{\SEenc}$ \\
    return $b'$ \\
    \\
    \underline{$\SEenc(M)$}\\[1pt]
    $IV \getsr \{0,1\}^n$ \\
    $M_1, \ldots, M_m \gets^{m} M$ \\
    for $i = 1$ to $m$ do \\
    \ind $P_i \getsr \{0,1\}^n$ \\
    \ind if $\TabT[IV \bigoplus \langle i \rangle] \neq \bot]$ then \\
    \ind\ind $\badtrue$ \\
    \ind\ind ${\color{blue} P_i \gets \TabT[IV \bigoplus \langle i \rangle]}$ \\
    \ind $\TabT[IV \bigoplus \langle i \rangle] \gets P_i$ \\
    \ind $C_i \gets P_i \bigoplus M_i$ \\
    return $IV \parallel C_1 \parallel \cdots \parallel C_m$
	}
	\fpage{.25}{
    \underline{G4}\\[1pt]
    $b' \gets \advA^{\SEenc}$ \\
    return $b'$ \\
    \\
    \underline{$\SEenc(M)$}\\[1pt]
    $IV \getsr \{0,1\}^n$ \\
    $M_1, \ldots, M_m \gets^{m} M$ \\
    for $i = 1$ to $m$ do \\
    \ind if $\TabT[IV \bigoplus \langle i \rangle] \neq \bot]$ then \\
    \ind\ind $\badtrue$ \\
    \ind $\TabT[IV \bigoplus \langle i \rangle] \gets 1$ \\
    \ind $C_i \getsr \{0,1\}^n$ \\
    return $IV \parallel C_1 \parallel \cdots \parallel C_m$
	}
  \caption{Game-hopping argument for proving \INDRAND for CTR-mode}
  \label{fig:indrand-ctr}
\end{figure}


\begin{theorem}
Let $\CTR$ be the CTR-mode symmetric encryption scheme using blockcipher
$\varepsilon$.  Let $\advA$ be an $\INDRAND_{\CTR}$-adversary making at most
$q$ queries each totaling at most $\sigma$ blocks. Then we give an
$\textup{PRF}_{\varepsilon}$-adversary $\advB$ such that

$$
  \adv^{\INDRAND}_{\CTR}(\advA) \leq \adv^{\textup{prf}}_{\varepsilon}(\advB) + 2 \sigma q^2 / 2^n
$$
\end{theorem}

\paragraph{Proof.}

The proof is a game-hopping argument, as seen in \cref{fig:indrand-ctr}.
First, notice that $G0$ game is the same as the $\RORreal$ game for $\INDRAND$
security, and thus

$$
\Prob{\RORreal^{\advA}_{\CTR} \Rightarrow 1} = \Prob{G0 \Rightarrow 1}
$$

Next, notice that $G0$ corresponds to $\textup{PRF1}$ and $G1$ corresponds to
$\textup{PRF0}$. So

\begin{align*}
  \absv{\Prob{G0 \Rightarrow 1} - \Prob{G1 \Rightarrow 1}} = \adv^{\textup{prf}}_{\varepsilon}(\advB) \\
  \Prob{G0 \Rightarrow 1} \leq \Prob{G1 \Rightarrow 1} + \adv^{\textup{prf}}_{\varepsilon}(\advB) \\
\end{align*}

Next, since in $G2$ the value in $\TabT$ is re-used in the event of a collision,
it essentially acts a random function, which implies
$\Prob{G1 \Rightarrow 1} = \Prob{G2 \Rightarrow 1}$.

Since $G2$ and $G3$ are identical-until-bad, by the fundamental lemma of
game playing we have

$$
\Prob{G2 \Rightarrow 1} \leq \Prob{G3 \Rightarrow 1} + \Prob{\bad_3}
$$

The only difference between $G3$ and $G4$ is that $G3$ derives ciphertext
using one-time pads and $G4$ derives ciphertext from random strings. But being
a $\INDCPA$-adversary, $\advA$ should not be able to distinguish one from the
other. So $\Prob{G3 \Rightarrow 1} = \Prob{G4 \Rightarrow 1}$.
Furthermore, notice that $\bad$ is set in the same conditions as $G3$ and $G4$,
so $\Prob{\bad_3} = \Prob{\bad_4}$. Then

$$
\Prob{G3 \Rightarrow 1} + \Prob{\bad_3} = \Prob{G4 \Rightarrow 1} + \Prob{\bad_4}
$$

Next, since $G4$ assigns the ciphertext to random bitstrings,

$$
\Prob{G4 \Rightarrow 1} = \Prob{\RORrand^{\advA}_{\CTR} \Rightarrow 1}
$$

Finally, let $b_1, \ldots, b_q$ be the number of blocks in each query by
$\advA$, where $\forall 1 \leq i \leq q. b_i \leq \sigma$.
We use the argument from Lemma 5.18 in
Bellare-Rogaway to argue the bound on $\Prob{\bad_4}$ below.  The argument
rests on the following problem: given a table of numbers whose items are drawn
from $\{0,1\}^n$, where there are $q$ rows in the table and row $i$ has $b_i$
columns, what is the probability that a collision among the entries occur?
Lemma 5.18 gives the probability $\texttt{Col}$ of
collision as

$$
\Prob{\texttt{Col}} \leq \frac{(q-1)(\sum^q b_i)}{2^n}
$$

Notice that the entries of this table correspond to
$IV \bigoplus \langle i \rangle$ in $G4$,
and a collision in the table corresponds to $\bad_4$ occuring. Thus

$$
\Prob{\texttt{Col}} = \Prob{\bad_4} \leq
\frac{(q_1)(\sum^q b_i)}{2^n} \leq
\frac{(q-1) q \sigma}{2^n} =
\frac{\sigma (q^2 -q)}{2^n} \leq
\frac{2 \sigma q^2}{2^n}
$$

Putting all of these facts together, we have

\begin{align*}
  \adv^{\INDRAND}_{\CTR}(\advA)
  &= \absv{\Prob{\RORreal^{\advA}_{\CTR} \Rightarrow 1} - \Prob{\RORrand^{\advA}_{\CTR}} \Rightarrow 1} \\
  &= \absv{\Prob{G0 \Rightarrow 1} - \Prob{\RORrand^{\advA}_{\CTR} \Rightarrow 1}} \\
  &\leq \absv{\Prob{G1 \Rightarrow 1} + \adv^{\textup{prf}}_{\varepsilon}(\advB) - \Prob{\RORrand^{\advA}_{\CTR} \Rightarrow 1}} \\
  &= \absv{\Prob{G2 \Rightarrow 1} + \adv^{\textup{prf}}_{\varepsilon}(\advB) - \Prob{\RORrand^{\advA}_{\CTR} \Rightarrow 1}} \\
  &\leq \absv{\Prob{G3 \Rightarrow 1} + \Prob{\bad_3} + \adv^{\textup{prf}}(\advB) - \Prob{\RORrand^{\advA}_{\CTR} \Rightarrow 1}} \\
  &= \absv{\Prob{G4 \Rightarrow 1} + \Prob{\bad_4} + \adv^{\textup{prf}}_{\varepsilon}(\advB) - \Prob{\RORrand^{\advA}_{\CTR} \Rightarrow 1}} \\
  &= \Prob{\bad_4} + \adv^{\textup{prf}}_{\varepsilon}(\advB) \\
  &\leq \adv^{\textup{prf}}_{\varepsilon}(\advB) + 2 \sigma q^2 / 2^n
\end{align*}

as needed. $\blacksquare$

