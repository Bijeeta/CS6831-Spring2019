%!TEX root = ../main.tex
%%%%%%%%%%%%%%%%%%%%%%%%%%%%%%%%%%%%%%%%%%%%%%%%%%%%%%%%%%%%%%%%%%%%%%%%%%%%%%%%
\subsection{Hash-based PRF and the Random Oracle Model}

\begin{wrapfigure}{r}{2in}
  \centering
\begin{tikzpicture}[scale=0.4]
	\node[draw,trapezium,trapezium left angle=70,trapezium right angle=70,minimum height=1.0cm,thick,shift={(1.15,0)},rotate=-90] (hfxn)
	{\begin{sideways}\Large $\hash$ \end{sideways}};
	\draw[->,thick] (0,0) node[left] {$\prfkey \concat \msg$} -- (hfxn);
	\draw[->,thick] (hfxn) -- ++(3,0) node[right] {$Y$};
\end{tikzpicture}
\caption{Simple hash-based PRF.}
\label{fig:hash-based-prf}
\end{wrapfigure}

Another property that is often desired of hash functions is that of a PRF.
Consider a simple hash-based PRF construction, $\prf:\keyspace\times\msgspace\rightarrow\bits^{n}$, constructed from a hash function $\hash:\bits^{*}\rightarrow\bits^{n}$ such that $\prf_{\prfkey}(\msg) = \hash(\prfkey\concat\msg)$.

Now we can consider how to prove the PRF security of $\prf$.
One approach would be to instantiate $\hash$ with a specific hash function construction, e.g. SHA-2 or SHA-3, and prove security with respect to the building blocks of those constructions, e.g. PRF security of the compression function or PRP security of the permutation.
However, this would not be a modular approach, as a new proof would be required for each hash function.
Instead, we will use a similar approach to as when we proved the PRF security of the Davies-Meyer compression function.
\scribenote{Add reference}
There we introduced an idealized model for the block cipher, called the ideal cipher model.
Here, we will introduce another idealized model known as the \emph{random oracle model} (ROM), in which we, as the name suggests, model the hash function as a random oracle.

\begin{wrapfigure}{r}{2in}
  \centering
\hfpages{.15}{
\underline{$\PRF1_{\prf,\Horacle}^\advA$}\vspace{0.5em}\\
$K \getsr \keyspace$\\
$b' \getsr \advA^{\Fn,\Horacle}$\\
Return $b'$\medskip

\underline{$\Fn(M)$}\\
Return $F^\Horacle_K(M)$\medskip

\underline{$\Horacle(X)$}\\
If $\TabH[X] = \bot$ then\\
\myInd $\TabH[X] \getsr \bits^n$\\
Return $\TabH[X]$

}{
\underline{$\PRF0_{F,\Horacle}^\advA$}\vspace{0.3em}\\
$\rho \getsr \Func(\msgspace,n)$\\
$b' \getsr \advA^{\Fn,\Horacle}$\\
Return $b'$\medskip

\underline{$\Fn(M)$}\\
Return $\rho(M)$\\

\underline{$\Horacle(X)$}\\
If $\TabH[X] = \bot$ then\\
\myInd $\TabH[X] \getsr \bits^n$\\
Return $\TabH[X]$
}
\caption{PRF security games in the random oracle model.}
\label{fig:prf-ro}
\end{wrapfigure}

Consider the modified PRF security games shown in Figure~\ref{fig:prf-ro}.
In the ROM, a new oracle $\Horacle$ is added to the security game.
Calling the hash function is done by sending queries to $\Horacle$.
The random oracle $\Horacle$ is instantiated lazily by maintaining a table $\TabH$, randomly sampling and storing in $\TabH$ return values for new queries.
We prove the following theorem about the security of the hash-based PRF in the random oracle model:

\begin{theorem}
Let  $\Horacle\Colon\msgspace\rightarrow\bits^n$ be modeled as a random oracle
and let $F^\Horacle\Colon\keyspace\times\msgspace\rightarrow\bits^n$ be the
hash-based PRF defined as $F^\Horacle(K,M) = \Horacle(K \concat M)$.
Then for any $\PRF_{F,\Horacle}$-adversary $\advA$ making at most $q$ queries
to $\Horacle$ it holds that
\bnm
  \AdvPRF{F,\Horacle}{\advA} \le \frac{q}{|\keyspace|} \;.
\enm
\end{theorem}


\begin{wrapfigure}{r}{3.8in}
  \centering
\hfpagesss{.15}{.18}{.15}{
\underline{$\G0$}\vspace{0.5em}\\
$\prfkey \getsr \keyspace$\\
$b' \getsr \advA^{\Fn,\Horacle}$\\
Return $b'$\medskip

\underline{$\Fn(\msg)$}\\
Return $\Horacle(\prfkey\concat\msg)$\medskip

\underline{$\Horacle(X)$}\\
If $\TabH[X] = \bot$ then\\
\myInd $\TabH[X] \getsr \bits^n$\\
Return $\TabH[X]$

}{
\underline{$\fbox{\G1}$\;\;$\G2$}\vspace{0.5em}\\
$\prfkey \getsr \keyspace$\\
$b' \getsr \advA^{\Fn,\Horacle}$\\
Return $b'$\medskip

\underline{$\Fn(\msg)$}\\
If $\TabT[\prfkey\concat\msg] = \bot$ then\\
\myInd $\TabT[\prfkey\concat\msg]\getsr \bits^{n}$\\
Return $\TabT[\prfkey\concat\msg]$\medskip

\underline{$\Horacle(X)$}\\
If $\TabT[X]\neq\bot$ then\\
\myInd $\badtrue$\\
\myInd \fbox{$\TabH[X]\gets\TabT[X]$}\\
If $\TabH[X] = \bot$ then\\
\myInd $\TabH[X] \getsr \bits^n$\\
Return $\TabH[X]$

}{
\underline{$\G3$}\vspace{0.3em}\\
$\rho \getsr \Func(\msgspace,n)$\\
$b' \getsr \advA^{\Fn,\Horacle}$\\
Return $b'$\medskip

\underline{$\Fn(M)$}\\
Return $\rho(M)$\\

\underline{$\Horacle(X)$}\\
If $\TabH[X] = \bot$ then\\
\myInd $\TabH[X] \getsr \bits^n$\\
Return $\TabH[X]$
}
\caption{PRF security games in the random oracle model.}
\label{fig:hash-prf-games}
\end{wrapfigure}

\emph{Proof.}
Consider the set of games given in Figure~\ref{fig:hash-prf-games}.
Game \G0 is identical to $\PRF1_{\prf,\Horacle}$ with pseudocode for the PRF oracle $\Fn$ filled in to represent the hash-based PRF.
In Game \G1, the PRF oracle does not call the random oracle, but instead maintains a separate table $\TabT$ to respond to queries with random values.
If the random oracle $\Horacle$ is called with a key\dash message pair that is stored in $\TabT$, a $\bad$ flag is set and the random value sampled in $\Fn$ is copied over and $\Horacle$ responds in a consistent manner.
Since the value is copied over from $\TabT$ to $\TabH$, \G1 is identical to \G0.

Game \G2 is identical until $\bad$ to \G1.
Instead of copying the value from $\TabT$ to $\TabH$, a new random value is sampled.
By the fundamental lemma of game playing, we bound the difference in advantage between \G1 and \G2 by the probability the $\bad$ flag is set.
To set the $\bad$ flag, $\advA$ must make a query to $\Horacle$ that begins with PRF key $\prfkey$.
Since only random values are returned, $\advA$ has no knowledge of $\prfkey$, so on any given query, the probability the $\bad$ flag is set is $1/\abs{\keyspace}$.
Thus, over $q$ queries, using the union bound we have
\begin{align*}
	\absv*{\Prob{\G1\Rightarrow 1} - \Prob{\G2\Rightarrow1}} &\le \Prob{\G1 \setsbad}\\
  &\le \frac{q}{\abs{\keyspace}}.
\end{align*}

Lastly, \G2 is identical to \G3 is identical to $\PRF0_{\prf,\Horacle}$.
Since the values $\TabH$ is not updated with the values from $\TabT$ both $\Fn$ and $\Horacle$ act as independent random functions.\hfill$\dqed$

So we've proved security of the hash-based PRF security when the hash function is modeled by a random oracle.
However, how do we gain confidence that the random oracle model fits for specific hash functions?
In other words, if we prove a protocol secure with respect to the random oracle model, is it safe to instantiate the random oracle with a hash function such as SHA-2 or SHA-3?
In the case of the ideal cipher model, we gain confidence through use of cryptanalysis of specific block ciphers.

\begin{wrapfigure}{r}{2.9in}
\begin{tikzpicture}[scale=0.4]

	\begin{scope}[]
		\node [draw,trapezium,trapezium left angle=50,trapezium right angle=90,minimum height=0.75cm,thick,shift={(1.15,0.3)},rotate=-90]
		{\begin{sideways}\Large$f$\end{sideways}};
		\draw[->,thick] ++(0.5,+4) node[above] {$\prfkey$} -- ++(0,-1.5) -- ++(1.4,0);
		\draw[->,thick] ++(0,0.5) node[left] {$\IV$} -- ++(1.9,0);
	\end{scope}

	\begin{scope}[shift={(3.8,0)}]
		\node [draw,trapezium,trapezium left angle=50,trapezium right angle=90,minimum height=0.75cm,thick,shift={(1.35,0.3)},rotate=-90] (centerbox)
		{\begin{sideways}\Large$f$\end{sideways}};
		\draw[->,thick] ++(0.9,+4) node[above] {$\msg_1$} -- ++(0,-1.5) -- ++(1.4,0);
		\draw[->,thick] ++(0,0.5) -- ++(2.4,0);
	\end{scope}

	\begin{scope}[shift={(8.2,0)}]
		\node [draw,trapezium,trapezium left angle=50,trapezium right angle=90,minimum height=0.75cm,thick,shift={(1.35,0.3)},rotate=-90]
		{\begin{sideways}\Large$f$\end{sideways}};
		\draw[->,thick] ++(0.9,+4) node[above] {$\qquad\msg_2 \concat 10^r \concat \bm{\langle} \left| \msg \right| \bm{\rangle}$} -- ++(0,-1.5) -- ++(1.4,0);
		\draw[->,thick] ++(0,0.5) -- ++(2.4,0);
		\draw[->,thick] ++(4.4,0.5) -- ++(2,0) node[right] {$Y$};
	\end{scope}

	\begin{scope}[shift={(7, -9)}]
		\node [draw,trapezium,trapezium left angle=50,trapezium right angle=90,minimum height=0.75cm,thick,shift={(1.15,0.3)},rotate=-90]
		{\begin{sideways}\Large$f$\end{sideways}};
		\draw[->,thick] ++(0.5,+4) node[above] {$\qquad\msg^* \concat 10^{r'} \concat \bm{\langle} \left| \msg' \right| \bm{\rangle}$} -- ++(0,-1.5) -- ++(1.4,0);
		\draw[->,thick] ++(4.2,0.5) -- ++(2,0) node[right] {$Y'$};
	\end{scope}

	\draw[->,thick] (16,0.5) -- ++(1,0) -- ++(0, -3) -- ++(-15, 0) -- ++(0, -6) -- ++(6.9,0);
	\path (2, -11) node[anchor=west] {$\msg = \msg_1\concat\msg_2$};
	\path (1.8, -12.5) node[anchor=west] {$\msg' = \msg_1\concat\msg_2\concat 10^r\concat\bm{\langle}\left|\msg\right|\bm{\rangle}\concat\msg^*$};

\end{tikzpicture}
\caption{Length extension attack when instantiating the proposed hash-based PRF construction with a Merkle-Damg\aa rd hash function.}
\label{fig:length-extension-md}
\end{wrapfigure}

In fact, this can be a problem in practice, as real hash functions often do not act as random oracles, i.e., there exist attacks that take advantage of the structure of the hash function.
Take the example of instantiating the hash-based PRF that we just proved secure in the random oracle model with the Merkle-Damg\aa rd function.
This construction is distinguishable from a PRF through a straightforward length extension attack.
The attack is depicted in Figure~\ref{fig:length-extension-md}.
The output of the Merkle\dash Damg\aa rd hash function on one message can be chained into computing the output for a longer message without knowledge of the secret key.

There are variants of the basic Merkle\dash Damg\aa rd hash function that can be used to instantiate the hash-based construction, but this leads us to a larger question.
If not all hash functions are properly modeled by the random oracle model, how do we proceed?
As mentioned before, a non-modular approach would be to prove protocols without the random oracle model.
Another approach is to build hash functions that are better modeled by the random oracle model.
This is a notion known as indifferentiability and we will look at more closely later in the section.
One of the strengths of the sponge construction from Figure~\ref{fig:sponge} is that it meets this indifferentiability property.
