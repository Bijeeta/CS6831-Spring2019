%!TEX root = ../main.tex
%%%%%%%%%%%%%%%%%%%%%%%%%%%%%%%%%%%%%%%%%%%%%%%%%%%%%%%%%%%%%%%%%%%%%%%%%%%%%%%%
\section{Further Security Properties for Hash Functions}

Hash functions are used to accomplish a wide array tasks in cryptography that require properties stronger than, weaker than, and orthogonal to just collision resistance.
In this section, we will examine a few of these alternate security goals for hash functions and their relations to each other.

\subsection{Preimage and Second Preimage Resistance}

Preimage resistance, often known as ``one-wayness'' of a one-way function, means that given an output $Y$ of a function $\hash$, it is hard to invert $\hash$ and find an element of the domain space that evaluates to $Y$, i.e., it is hard to find any preimage of $Y$.
Second preimage resistance means that given a domain element and function $\hash$, it is hard to find a second domain element that evaluates to the same output $Y$ for both domain elements, i.e., it is hard to find a second preimage of $Y$ even if already given one preimage.
Pseudocode for preimage resistance and second preimage resistance, $\OWF$ and $\SPR$, respectively, is given in Figure~\ref{fig:preimage-resistance}.
The advantage of an adversary against these games is defined as:
\bnm
\AdvOWF{\hash}{\advA} = \Prob{\OWF^\advA_{\hash}\Rightarrow\true}\hspace{2em}\text{and}\hspace{2em}\AdvSPR{\hash}{\advA} = \Prob{\SPR^\advA_{\hash}\Rightarrow\true}\;.
\enm

\begin{wrapfigure}{r}{3.5in}
  \centering
\hfpagess{.2}{.3}{
\underline{$\OWF^\advA_{\hash}$}\\[1pt]
$M \getsr \msgspace$\\
$Y \gets \hash(M)$\\
$M' \getsr \advA(Y)$\\
Return $(\hash(M') = Y)$
}{
\underline{$\SPR^\advA_{\hash}$}\\[1pt]
$M \getsr \msgspace$\\
$Y \gets \hash(M)$\\
$M' \getsr \advA(M)$\\
Return $(\hash(M') = \hash(M)) \land (M \ne M')$
}
\caption{
Security games for Preimage resistance (left) and second preimage resistance (right).
}
\label{fig:preimage-resistance}
\end{wrapfigure}

These definitions, along with collision resistance, are certainly closely related.
We can think about how they compare to each other.
For example, we might conjecture $\CR\Rightarrow\SPR$.
Thinking about the contrapositive, if one can find a second preimage of an output $Y$ than those two preimages constitute a collision.
It turns out that we can formally show the following relations: $\CR\Rightarrow\SPR\Rightarrow\OWF$;
and furthermore, there exist separations for the opposite direction: $\OWF\not\Rightarrow\SPR\not\Rightarrow\CR$.
Formally showing these implications and separations is left as an exercise.
Note that these results hold specifically for the definitions of preimage resistance, second preimage resistance, and collision resistance given in this book.
While they are perhaps the most common such definitions, there exist several nuanced alternate definitions for these notions in the literature that do not satisfy the above implications.
Rogaway and Shrimpton~\cite{RogawayShrimpton04} explore the landscape of these definitions.
