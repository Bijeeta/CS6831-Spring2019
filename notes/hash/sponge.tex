%!TEX root = ../main.tex
%%%%%%%%%%%%%%%%%%%%%%%%%%%%%%%%%%%%%%%%%%%%%%%%%%%%%%%%%%%%%%%%%%%%%%%%%%%%%%%%
\subsection{Sponge Construction}
We've seen one popular approach to building hash functions, namely using the Merkle-Damg\aa rd domain extension transform with a compression function such as Davies-Meyer.
It is the approach used by many standardized hash functions (MD5, SHA-1, SHA-2).
In this section, we will look at an alternate design approach to building hash functions called the sponge construction, which is used in the standardized SHA-3 hash function.
The sponge construction is illustrated in Figure~\ref{fig:sponge}.

The sponge construction maintains an internal state of length $r+c$ bits known as ``rate'' bits ($r$) and ``capacity'' bits ($c$).
It also makes use of a permutation $f$ on domain $\bits^{r+c}$.
The sponge construction is executed in two phases, the ``absorbing'' phase and the ``squeezing'' phase.
One round of the absorbing phase corresponds to XORing a message block of length $r$ with the rate bits and then passing the entire internal state (rate and capacity bits) through permutation $f$ to get a new internal state for the next round.
The number of rounds, i.e. rate of absorbing, is determined by $r$, the size of blocks into which the message must be broken.
In the squeezing phase, $r$ bits of the hash digest are read off from the rate bits in each squeezing round.
Between squeezing rounds, the internal state is again passed through the permutation $f$.
A nice property that emerges from this approach is that the hash digest size can be chosen as a parameter indicating the number of squeezing rounds to perform.

We will see in the next section that Merkle-Damg\aa rd style constructions are subject to a few subtle ``gotchas'' if used incorrectly, e.g., malleability attacks.
While there exist fixes to the basic Merkle\dash Damg\aa rd construction to alleviate these concerns, many believe the sponge construction addresses them in a more elegant way leading to a cleaner analysis.
It turns out that the use of capacity bits as part of the internal state is critical to the security properties of the construction.
In particular, the capacity bits do not directly interact with inputs and are not directly returned as outputs.
In contrast, the hash digest of the Merkle\dash Damg\aa rd construction is the complete internal state of the function.


\begin{figure}[t]
  \centering
\begin{tikzpicture}[scale=0.45]

\tikzset{SpongePerm/.style=rounded corners=4pt,};
%\tikzset{edge/.style=-latex new, arrow head=8pt, thick};
%\tikzset{edgee/.style=latex new-latex new, arrow head=8pt, thick};
\tikzset{edge/.style=->, thick};
\tikzset{edgee/.style=<->, thick};

\path (12.5, -1) node {absorbing phase};
\path (21.5, -1) node {squeezing phase};

%%
\begin{scope}[xshift=0cm]
  \draw[thick] (0,0) rectangle node {$0$} ++(1,3);
  \draw[thick] (0,3) rectangle node {$0$} ++(1,7);

  \node[XOR,thick] (xm0) at (1+1.5,8) {};
  \draw[edge,thick] (1,8) -- (xm0);
  \draw[edge,thick] (1,2) -- ++(3,0);
  \draw[edge,thick] (1+1.5,10.5) node[above] {\large $\msg_{0}$} -- (xm0);
  \draw[edge,thick] (xm0) -- ++(1.5,0);

	\draw[edgee,anchor=east] (-1,0) -- node[left] {$c$ bits} ++(0,3);
	\draw[edgee,anchor=east] (-1,3) -- node[left] {$r$ bits} ++(0,7);

\end{scope}

\foreach \z in {1,2} {
  \begin{scope}[xshift=\z*4cm]
    \draw[SpongePerm]
    		(0,0) rectangle node {\large$f$} ++(1,10);
  \end{scope}
}

\foreach \z in {4,11} {
  \begin{scope}[xshift=\z*1cm]
    \node[XOR,thick] (xm\z) at (1+1.5,8) {};
    \draw[edge,thick] (1,8) -- (xm\z);
    \draw[edge,thick] (1,2) -- ++(3,0);
    \draw[edge,thick] (xm\z) -- ++(1.5,0);
  \end{scope}
}
\draw[edge,thick] (5+1.5,10.5) node[above] {\large $\msg_{1}$} -- (xm4);
\draw[edge,thick] (12+1.5,10.5) node[above] {\large $\msg_{n-1}$} -- (xm11);

\begin{scope}[xshift=8cm]
  \draw[edge,thick] (1,2) -- ++(1.5,0) node[right] {$\dots$};
  \draw[edge,thick] (1,8) -- ++(1.5,0) node[right] {$\dots$};
\end{scope}

%%
\begin{scope}[xshift=15cm]
  \draw[SpongePerm]
  		(0,0) rectangle node {\large$f$} ++(1,10);

  \draw[thick] (1,2) -- ++(1,0);
  \draw[thick] (1,8) -- ++(1,0);
  \draw[dashed,thick] (2,-2) -- ++(0,14);
\end{scope}
%%

\foreach \z in {0,1} {
  \begin{scope}[xshift=\z*4cm+17cm]
    \draw[edge,thick] (0,2) -- ++(3,0);
    \draw[edge,thick] (0,8) -- ++(3,0);
    \draw[edge,thick] (0+1.5,8) -- ++(0,2.5) node[above] {\large $Y_{\z}$};
    \draw[SpongePerm] (3,0) rectangle node {\large$f$} ++(1,10);
  \end{scope}
}

\begin{scope}[xshift=25cm]
  \draw[edge,thick] (0,2) -- ++(1.5,0) node[right] {$\dots$};
  \draw[edge,thick] (0,8) -- ++(1.5,0) node[right] {$\dots$};
  \draw[edge,thick] (3,2) -- ++(1.5,0);
  \draw[edge,thick] (3,8) -- ++(1.5,0);
  \draw[SpongePerm] (4.5,0) rectangle node {\large$f$} ++(1,10);
  \draw[edge,thick] (5.5,8) -- ++(1.5,0) -- ++(0,2.5) node[above] {\large $Y_{m-1}$};
\end{scope}
\end{tikzpicture}
\caption{The sponge construction for hash functions built from permutation $f$, similar to as used in SHA-3.
Variable size messages are broken into $n$ blocks of size $r$ and incorporated in $n$ absorbing rounds.
The hash digest of length $mr$ is read out in $m$ squeezing rounds.
}
\label{fig:sponge}
\end{figure}
