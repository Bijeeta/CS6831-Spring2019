%%%%%%%%%%%%%%%%%%%%%%%%%%%%%%%%%%%%%%%%%%%%%%%%%%%%%%%%%%%%%%%%%%%%%%%%%%%%%%%%
\section{Block Cipher Design and Cryptanalysis}
\label{sec:cryptanalysis}

We've seen some theoretical ways to build block ciphers, namely Feistel
constructions using round functions that are secure as PRFs. There are other
constructions as well, such as Evan-Mansour \tnote{Should probably include
somewhere}. But this is reductive, since it's unclear how to build the PRFs
themselves. One could imagine trying to use actual random functions, but this is
intractable for large block sizes --- the secret key in this case being a random
table requring $n2^n$ bits. 

In practice block ciphers are built using a bag of design principles that have been built
up over the last 60 or so years, in response to new techniques for
cryptanalysis.

\tnote{Perhaps discuss really trivially broken ciphers, such as ones that are
completely linear functions, or partially linear functions}


\paragraph{Linear cryptanalysis.} \tnote{Discuss Matsui's paper, the high level
intuition and how it can be used to break DES}  

\tnote{Also shown in Boneh-Shoup book}


\bne
\label{eq:linear-approx}
  \Prob{M[S_m] \oplus C[S_c] = K[S_k]} = \frac{1}{2} + \epsilon
\ene

\begin{theorem}
Let $\cipher$ be a cipher such that \eqref{eq:linear-approx} holds with
$\epsilon > 0$, and let $K \in \calK$. Let $M_1,\ldots,M_q$ be sampled uniformly
from $\bits^n$ and let $C_i = E_K(M_i)$ for $i \in \{1,\ldots,q\}$. Then 
\bnm
  \Prob{K[S_k] = \Maj\left(\{M_i[S_m]\oplus C_i[S_c]\}_{i=1}^q\right)} \ge 1 - e^{-q\epsilon^2/2} \;.
\enm
\end{theorem}


\begin{theorem}[Chernoff bounds]
Let $X = \sum_{i=1}^n X_i$, all $X_i$ independent and where $X_i = 1$ with probability $p_i$ and $X_i = 0$
with probability $1-p_i$. Let $\mu = \Ex[X]$. Then  
\begin{align}
  \Prob{X \ge (1 + \delta)\mu} &\le e^{-\frac{\delta^2}{2+\delta}\mu}\\
  \Prob{X \le (1 + \delta)\mu} &\le e^{-\frac{\delta^2}{2}\mu}
\end{align}
\end{theorem}


\begin{lemma}
Let $X_i$ for $1 \le i \le n$ be indendent random variables with probabilities
$p_i$ of being one and $1-p_i$ of being zero. Then  
\bnm
  \Prob{X_1\oplus \cdots \oplus X_n = 0} = \frac{1}{2} + 2^{n-1}\prod_{i=1}^n
  \left(p_i - \frac{1}{2}\right) \;.
\enm
\end{lemma}


\bnm
  \AdvPRF{\EM,P}{\advA} \le \frac{2\cdot q_1 \cdot q_2}{2^n}
\enm
