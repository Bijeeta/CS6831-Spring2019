%%%%%%%%%%%%%%%%%%%%%%%%%%%%%%%%%%%%%%%%%%%%%%%%%%%%%%%%%%%%%%%%%%%%%%%%%%%%%%%%
\section{Authenticated Encryption Variants}
\label{sec:authenc}

An authenticated encryption with associated data (AEAD) scheme is a triple of
algorithms $\AEAD = (\kg,\enc,\dec)$. It is the same as an SE scheme, except
that encryption and decryption both take an additional input, called the
associated data. Sometimes the associated data is referred to as the header.

\hfpagess{.15}{.15}{
\underline{$\RORCCA1^\advA_{\AEAD}$}\\[1pt]
$K \getsr \kg$\\
$b' \getsr \advA^{\EncOracle,\DecOracle}$\\
Ret $b'$\medskip

\underline{$\EncOracle(H,M)$}\\
$C \getsr \enc_K(H,M)$\\
$\calC \gets \calC \cup \{(H,C)\}$\\
Ret $C$\medskip

\underline{$\DecOracle(H,C)$}\\
If $(H,C) \in \calC$ then \\
\myInd Ret $\bot$\\
Ret $\dec_K(H,C)$
}{
\underline{$\RORCCA0^\advA_{\AEAD}$}\\[1pt]
$b' \getsr \advA^{\EncOracle,\DecOracle}$\\
Ret $b'$\medskip

\underline{$\EncOracle(H,M)$}\\
$C \getsr \bits^{\ctxtlen(|M|)}$\\
Ret $C$\medskip

\underline{$\DecOracle(H,C)$}\\
Ret $\bot$
}


\bnm
\AdvRORCCA{\SE}{\advA} = \left|\Prob{\RORCCA1_\SE^\advA\Rightarrow1}
                                    -\Prob{\RORCCA0_\SE^\advA\Rightarrow1}\right|
\enm


\fpage{.15}{
\underline{$\CTXT^\advA_{\SE}$}\\[1pt]
$K \getsr \kg$\\
$\win \gets \false$\\
$\advA^{\EncOracle,\DecOracle}$\\
Ret $\win$\medskip

\underline{$\EncOracle(M)$}\\
$C \getsr \enc_K(M)$\\
$\calC \gets \calC \cup \{C\}$\\
Ret $C$\medskip

\underline{$\DecOracle(C)$}\\
If $C \in \calC$ then \\
\myInd Ret $\bot$\\
$M \gets \dec_K(C)$\\
If $M \ne \bot$ then \\
\myInd $\win\gets\true$\\
Ret $M$
}

\bnm
\AdvCTXT{\SE}{\advA} = \Prob{\CTXT_\SE^\advA\Rightarrow\true}\\
\enm


\begin{theorem}
Let $\SE$ be a symmetric encryption scheme. Let $\advA$ be any
$\INDCPA_\SE$-adversary making at most $q$ queries. 
We give an $\ROR_\SE$-adversary $\advB$ such that
\bnm
  \AdvINDCPA{\SE}{\advA} \le 2\cdotsm\AdvROR{\SE}{\advB}
\enm
Adversary $\advB$ makes at most $q$ 
queries and runs in time that of $\advA$.
\end{theorem}

\tnote{We'll possibly introduce the following at some point, but didn't need it
here yet.}
We sometimes use $\bigO$ notation to hide small values that can be derived from
proofs, but don't matter to the interpretation of the theorem. If we were to do
an asymptotic treatment, this would correspond to hiding constants, hence the
abuse of notation.  Thus in above theorem we would replace $q+3$ with
$\bigO(q)$. 


\fpage{.20}{
\underline{$\advB^{\Enc}$}\\[1pt]
$b \getsr \bits$\\
$b' \getsr \advA^\EncSim$\\
If $(b = b')$ then Ret 1\\
Ret 0\medskip

\underline{$\EncSim(M_0,M_1)$}\\
Return $\Enc(M_b)$
}

\begin{align*}
\AdvROR{\SE}{\advB} 
    &= \left|\Prob{\ROR1_\SE^\advB\Rightarrow 1} -
                                \Prob{\ROR0_\SE^\advB\Rightarrow 1}\right|\\
    &= \left|\Prob{\INDCPA_\SE^\advA\Rightarrow\true} - \frac{1}{2}\right|\\
    &= \left|\frac{1}{2} +
    \frac{1}{2}\cdot\AdvINDCPA{\SE}{\advA} - \frac{1}{2}\right|\\
    &= \frac{1}{2}\cdot\AdvINDCPA{\SE}{\advA}
\end{align*}



\tnote{Let's add this strong tweakable block cipher stuff to the chapter on
tweakable block ciphers. The scribe for Feb 18 can write it up, but we'll merge
it into previous chapter later.}


\hfpages{.15}{
		\underline{$\STPRP1_{\tweakCipher}^\advA$}\\
		$K \getsr \keyspace$\\
		$b' \getsr \advA^{\Fn,\FnInv}$\\
		Return $b'$\medskip
		
		\underline{$\Fn(T,M)$}\\[1pt]
		Return $\tweakE_K(T,M)$\medskip
  
   \underline{$\FnInv(T,C)$}\\[1pt]
		Return $\tweakD_K(T,C)$

	}{
		\underline{$\STPRP0_{\tweakCipher}^\advA$}\\
		$\tweakpi \getsr \Perm(\tweakspace,\msgspace)$\\
		$b' \getsr \advA^{\Fn,\FnInv}$\\
		Return $b'$\medskip
		
		\underline{$\Fn(T,M)$}\\[1pt]
		Return $\tweakpi(T,M)$\medskip
  
   \underline{$\FnInv(T,C)$}\\[1pt]
		Return $\tweakpi^{-1}_K(T,C)$
	}

\bnm
\AdvSTPRP{\tweakCipher}{\advA} = \left|\Prob{\STPRP1^\advA\Rightarrow1} -
                                    \Prob{\STPRP0^\advA\Rightarrow1} \right|
\enm


